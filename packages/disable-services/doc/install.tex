% -*- latex -*-
%
% $Id: install.tex,v 1.3 2002/11/16 19:21:23 jsquyres Exp $
%
% $COPYRIGHT$
%

\subsection{Disabling Services}

The \cmd{disable-services} OSCAR package disables the following
services (if they exist) on the client nodes:

\begin{itemize}
\item Incoming mail service: the \cmd{sendmail}, \cmd{exim}, and
  \cmd{postfix} daemons are disabled.  This prevents incoming mail
  from being received on the nodes.  Note that outgoing mail is not
  disabled; most outgoing mail is sent immediately.  If there is some
  transient failure and the mail is not sent immediately, it will go
  to the mail service's queue.  A \file{cron.hourly} crontab is
  installed that runs every hour to trigger the queue in case this
  happens.

\item Kudzu: the Kudzu service looks for new hardware, usually upon
  boot up.  At boot time, this process takes over 30 seconds.  It is
  disabled in order to speed up the booting of individual nodes.
  
\item \cmd{slocate}: the \cmd{slocate} service runs a top-level
  \cmd{find} command over all local filesystems periodically (some
  Linux distributions have it set to run daily, others have it set to
  run weekly) in order to index all filenames for quick lookup using
  the \cmd{locate} command.  The top-level \cmd{find} command takes
  significant amounts of system resources to run, and is therefore
  disabled.
  
\item \cmd{makewhatis}: the \cmd{makewhatis} command is run via
  crontab (sometimes daily, sometimes weekly -- depending on the
  specific Linux distribution) to generate manual page indexes.  As
  with \cmd{slocate}, this command takes significant amounts of system
  resources to run, and is therefore disabled.
\end{itemize}

Note that these services are not uninstalled -- they are simply
disabled.  Administrators are free to re-enable them if they wish.
