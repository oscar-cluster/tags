% -*- latex -*-
%
% Copyright (c) 2002 The Trustees of Indiana University.  
%                    All rights reserved.
%
% This file is part of the Env-switcher software package.  For license
% information, see the COPYING file in the top-level directory of the
% Env-switcher source distribution.
%
% $Id: common.tex,v 1.14 2004/03/09 04:23:33 jsquyres Exp $
%

OSCAR has a generalized mechanism to both set a system-level default
MPI implementation, and also to allow users to override the
system-level default with their own choice of MPI implementation.

This allows multiple MPI implementations to be installed on an OSCAR
cluster (e.g., LAM/MPI and MPICH), yet still provide unambiguous MPI
implementation selection for each user such that ``\cmd{mpicc foo.c -o
  foo}'' will give deterministic results.

%%%%%%%%%%%%%%%%%%%%%%%%%%%%%%%%%%%%%%%%%%%%%%%%%%%%%%%%%%%%%%%%%%%%%%%%%%

\subsubsection{Setting the system-level default}

The system-level default MPI implementation can be set in two
different (yet equivalent) ways:

\begin{enumerate}
\item During the OSCAR installation, the GUI will prompt asking which
  MPI should be the system-level default.  This will set the default
  for all users on the system who do not provide their own individual
  MPI settings.
  
\item As \user{root}, execute the command:

\begin{verbatim}
  # switcher mpi --list
\end{verbatim}

   This will list all the MPI implementations available.  To set the
   system-level default, execute the command:

\begin{verbatim}
  # switcher mpi = name --system
\end{verbatim}
   
   where ``name'' is one of the names from the output of the
   \cmd{--list} command.
\end{enumerate}

{\bf NOTE:} System-level defaults for \cmd{switcher} are currently
propogated to the nodes on a periodic basis.  If you set the
system-level MPI default, you will either need to wait until the next
automatic ``push'' of configuration information, or manually execute
the \cmd{/opt/sync\_files/bin/sync\_files} command to push the changes to
the compute nodes.

{\bf NOTE:} Using the \cmd{switcher} command to change the default MPI
implementation will modify the \cmd{PATH} and \cmd{MANPATH} for all
{\em future} shell invocations -- it does {\em not} change the
environment of the shell in which it was invoked.  For example:

\begin{verbatim}
  # which mpicc
  /opt/lam-1.2.3/bin/mpicc
  # switcher mpi = mpich-4.5.6 --system
  # which mpicc
  /opt/lam-1.2.3/bin/mpicc
  # bash
  # which mpicc
  /opt/mpich-4.5.6/bin/mpicc
\end{verbatim}

If you wish to have your current shell reflect the status of your
switcher settings, you must run the ``\cmd{switcher-reload}''
command.  For example:

\begin{verbatim}
  # which mpicc
  /opt/lam-1.2.3/bin/mpicc
  # switcher mpi = mpich-4.5.6 --system
  # which mpicc
  /opt/lam-1.2.3/bin/mpicc
  # switcher-reload
  # which mpicc
  /opt/mpich-4.5.6/bin/mpicc
\end{verbatim}

Note that this is {\em only} necessary if you want to change your
current environment.  All new shells (including scripts) will
automatically get the new switcher settings.

%%%%%%%%%%%%%%%%%%%%%%%%%%%%%%%%%%%%%%%%%%%%%%%%%%%%%%%%%%%%%%%%%%%%%%%%%%

\subsubsection{Setting the user-level default}

Setting a user-level default is essentially the same as setting the
system-level default, except without the \cmd{--system} argument.
This will set the user-level default instead of the system-level
default:

\begin{verbatim}
  $ switcher mpi = lam-1.2.3
\end{verbatim}

% Stupid emacs latex mode: $

Using the special name \cmd{none} will indicate that no module should
be loaded for the \cmd{mpi} tag.  It is most often used by users to
specify that they do not want a particular software package loaded.

\begin{verbatim}
  $ switcher mpi = none
\end{verbatim}

% Stupid emacs latex mode: $

Removing a user default (and therefore reverting to the system-level
default) is done by removing the \cmd{default} attribute:

\begin{verbatim}
  $ switcher mpi --show
  user:default=mpich-1.2.4
  system:exists=true
  $ switcher mpi --rm-attr default
  $ switcher mpi --show
  system:default=lam-6.5.6
  system:exists=true
\end{verbatim}

% Stupid emacs latex mode: $
