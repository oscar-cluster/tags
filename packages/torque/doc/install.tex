% -*- latex -*-
%
% $Id: install.tex,v 1.10 2003/04/21 21:58:41 brechin Exp $
%
% $COPYRIGHT$
%

\subsection{Torque Resource Manager and Maui Scheduler}
\label{app:torque-overview}

Torque serves as the job launcher and batch queueing system for
OSCAR. Torque includes a basic FIFO scheduler with it, but is
disabled by default in OSCAR.  A more robust, open source scheduler
named ``Maui'' is used instead.

Basic Torque functionality is tested by OSCAR's test suite, and is
also used to launch jobs when testing other software included in
OSCAR. If the Torque test passes, Torque and Maui are up and
working.

If the users of your OSCAR cluster have not used Torque before, you
can expect somewhat of a learning curve.  The OSCAR user's
documentation contains some useful information to get them started.
The user's document instructs users to ask the system administrator
to provide them with sample Torque scripts used in the OSCAR test
suite.  Once the OSCAR test suite has been run (step 6 in the
install process), these scripts can be found in the home directory
of the \user{oscartst} user if the tests have been run previously.

\subsubsection{Configuring Torque}

By default, Torque installs without any queues or cluster specific
paremeters defined.  OSCAR configures Torque with sensible defaults
based on what it finds in the SIS database.  When the ``Complete
Cluster Setup'' step is executed from the wizard, the
\cmd{post\_install} script from the Torque package in OSCAR is
called. The \cmd{post\_install} configures only Torque parameters
that are non-existent, so as not to overwrite local customizations
by the system administrator.  However, if you wish to force the
default values back in place over any local customizations, the
\cmd{post\_install} script can be invoked manually with a
\cmd{--default} option.  This will revert all values to the original
OSCAR settings.

\cmd{qmgr} can be used to configure queues and Torque server
parameters. The OSCAR Torque \cmd{post\_install} script (located off
the top-level OSCAR installation directory in
\file{packages/torque/scripts}) uses \cmd{qmgr} behind the scenes.
There are man pages available, but reading the Torque admin guide is
the best way to learn how to use it. It is available on Torque's
homepage, listed below.  You will have to create an account on their
site in order to download the admin guide.

\subsubsection{Torque Resources}

Arbitrary node properties can be set by the administrator.  Torque
calls these properties ``resources''.  These resources can be
specified on the \cmd{qsub} command line when a user submits a job.
This allows a user to restrict their jobs to run only on nodes
exhibiting certain properties.  If some nodes of a cluster have more
memory, a different network, faster prcoessors, etc., jobs can be
submitted so they only run a specific subset.  These properties are
stored in plain text in \file{/usr/spool/PBS/server\_priv/nodes}.
However, if adjusted in the plain text file, the Torque server must
be restarted in order for changes to take effect.  The more
elaborate method is to use the \cmd{qmgr} command to modify node
properties via the Torque API.  OSCAR gives each node a starting
property of ``{\tt all}''.

\subsubsection{An FAQueue}

A popular misconception about Torque queues is that they are bound
to a group of nodes.  This is false.  If you have a four node queue
defined, it is not associated with any specific nodes.  You can
think of a queue as a multidimensional box that a job must fit in in
order to allow submission.  That is, the submitted parameters must
fit within certain max and min values for nodes, ppn (procs per
node), walltime, etc.  If specific nodes are desired to run on, then
resource attributes must be defined.

If you would like to get a full dump of your Torque server and queue
configuration, you can issue this command:

\begin{verbatim}
  # qmgr -c "print server"
\end{verbatim}

The \cmd{qmgr} interface can be used to define additional queues and
their parameters.  You can also change the parameters on the default
OSCAR queue, ``{\tt workq}''.  For example, to show the configuration
of the {\tt workq}, execute the following:

\begin{verbatim}
  # qmgr -c "list queue workq"
\end{verbatim}

To change any of the values listed, use the following:

\begin{verbatim}
  # qmgr -c "set queue workq PARAMETER = VALUE"
\end{verbatim}

where \cmd{PARAMETER} is a parameter from the ``\cmd{list queue}''
command, and \cmd{VALUE} is a valid value for that parameter.  You can
use the ``\cmd{print server}'' and/or ``\cmd{list queue}'' commands to
verify your changes.

Be aware that if you call the \cmd{post\_install} command with the
\cmd{--default} option, you will lose your customizations.  Also note
that OSCAR's default wallclock limit on {\tt workq} is 10,000 hours.
Depending on the application mix that will run on your cluster, you
may wish to adjust this value.

Some useful links:

\begin{itemize}
\item Torque: \url{http://www.clusterresources.com/products/torque/}

\item Maui Scheduler: \url{http://www.clusterresources.com/products/maui/}

\item OpenPBS: \url{http://www.openpbs.org/}

\item PBSPro: \url{http://www.pbspro.com/}
\end{itemize}

% LocalWords:  Exp
