% -*- latex -*-
%
% $Id: user.tex,v 1.12 2002/12/13 01:31:12 sad Exp $
%
% $COPYRIGHT$
%

\subsection{An overview of \cmd{ganglia}}
\label{app:ganglia-overview}

Ganglia is a real time cluster monitoring tool that uses a protocol
based on XML and XDR across a multicast network to provide a scalable
core tool set.  The cluster monitoring data can be accessed via a web
browser or via the command line. The core Ganglia monitoring toolkit
also includes the ability to add user defined data to the XML streams.

Ganglia was developed at the University of California, Berkeley
Computer Science Division as part of the ongoing clustering research
project named The Millennium Project (and its predecessor The NOW
Project).  It is being developed by Matt Massie
$<$\mailto{massie@cs.berkeley.edu}$>$, Brent Chun
$<$\mailto{bnc@caltech.edu}$>$, Steven Wagner $<$\mailto{swagner@ilm.com}$>$,
Federico Sacerdoti $<$\mailto{fds@sdsc.edu}$>$, and other active developers.
For additional information about the Ganglia project, see:

\vspace{10pt}
\centerline{\url{http://ganglia.sourceforge.net/}}
\vspace{10pt}

\rpmname{ganglia-webfrontend} is a package in OSCAR that sets up
the web page display on the head node of the data gathered by the
gmond and gmetad tools in the \rpmname{ganglia-monitor-core}.
This uses \cmd{rrdtool} package included with the OSCAR tool suite
as well as the \rpmname{apache}, \rpmname{php}, and \rpmname{expat}
packages installed with the core Linux distribution on the head node
to build an interactive useful display of the XML/XDR data being
passed across the multicast channel.

Once the cluster head node has the OSCAR tools installed a web browser
can be pointed at:

\vspace{10pt}
\centerline{\url{http://localhost/ganglia/}}
\vspace{10pt}

\noindent to see the ganglia web output. If nothing appears, check to
see if \cmd{apache} has been started along with the \cmd{gmond} and
\cmd{gmetad} processes. The start-up init scripts can be found
in \file{/etc/init.d/}.

The 2.0 version of OSCAR also includes the \cmd{ganglia} python
command line interface so access to the ganglia XML/XDR data can be
scripted. This is a tool contributed to the Ganglia project by the
Rocks team from the San Diego Supercomputer Center. Since it is based
on a python class file this will depend on having \rpmname{python}
and the \rpmname{PyXML-oscar} RPMs installed on the head node.

The current version of the Ganglia tools packaged for OSCAR is:

\begin{itemize}
\item \rpmname{ganglia-monitor-core-gmond-2.5.1-1oscar}
\item \rpmname{ganglia-monitor-core-gmetad-2.5.1-1oscar}
\item \rpmname{ganglia-monitor-core-lib-2.5.1-1oscar}
\item \rpmname{ganglia-webfrontend-2.5.1-1oscar}
\item \rpmname{ganglia-python-oscar-2.3-1}
\item \rpmname{rrdtool-oscar-1.0.35-3}
\item \rpmname{rrdtool-oscar-devel-1.0.35-3}
\item \rpmname{PyXML-oscar-0.7.1-7}
\end{itemize}

Once the ganglia-monitor-core-gmond rpm is installed there are some
user customizations that should be made in the gmond config file
located in:

\vspace{10pt}
\centerline{\cmd{/etc/gmond.conf}}
\vspace{10pt}

The file contains the following lines:

\begin{verbatim}

# This is the configuration file for the Ganglia Monitor Daemon (gmond)
# Documentation can be found at http://ganglia.sourceforge.net/docs/
#
# To change a value from it's default simply uncomment the line
# and alter the value
#####################
#
# The name of the cluster this node is a part of
# default: "unspecified"
# name  "My Cluster"
#
# The owner of this cluster. Represents an administrative
# domain. The pair name/owner should be unique for all clusters
# in the world.
# default: "unspecified"
# owner "My Organization"
#
# The latitude and longitude GPS coordinates of this cluster on earth.
# Specified to 1 mile accuracy with two decimal places per axis in Decimal
# DMS format: "N61.18 W130.50".
# default: "unspecified"
# latlong "N32.87 W117.22"
#
# The URL for more information on the Cluster. Intended to give purpose,
# owner, administration, and account details for this cluster.
# default: "unspecified"
# url "http://www.mycluster.edu/"
#
# The location of this host in the cluster. Given as a 3D coordinate:
# "Rack,Rank,Plane" that corresponds to a Euclidean coordinate "x,y,z".
# default: "unspecified"
# location "0,0,0"
#

\end{verbatim}

It is suggested that at least the name and owner configuration values
for the head node be set to unique values for your organization.

The addition of the \cmd{ganglia} command line tool makes it possible
to access all of the available multicasted ganglia monitoring data
that is flowing across the cluster from the command line or various
perl, shell, or other system admin created scripts.  The tool can be
found in:

\vspace{10pt}
\centerline{\cmd{/usr/sbin/ganglia}}
\vspace{10pt}

\noindent and the actual python class file gets installed in:

\vspace{10pt}
\centerline{\cmd{/usr/lib/python2.1/site-packages/gmon/ganglia.py}}
\vspace{10pt}

Running this tool on the command line gives the following usage
message:

\begin{verbatim}
  usage: ganglia < metric > [ metric metric ... ]
\end{verbatim}

where ``metric'' is one of:

\begin{itemize}
\item \cmdarg{sys\_clock}
\item \cmdarg{cpu\_nice}
\item \cmdarg{proc\_run}
\item \cmdarg{boottime}
\item \cmdarg{cpu\_system}
\item \cmdarg{mem\_shared}
\item \cmdarg{os\_release}
\item \cmdarg{cpu\_idle}
\item \cmdarg{load\_one}
\item \cmdarg{swap\_total}
\item \cmdarg{mem\_buffers}
\item \cmdarg{mem\_cached}
\item \cmdarg{proc\_total}
\item \cmdarg{os\_name}
\item \cmdarg{cpu\_speed}
\item \cmdarg{machine\_type}
\item \cmdarg{mem\_total}
\item \cmdarg{load\_five}
\item \cmdarg{load\_fifteen}
\item \cmdarg{cpu\_user}
\item \cmdarg{swap\_free}
\item \cmdarg{cpu\_num}
\item \cmdarg{cpu\_idle}
\item \cmdarg{mem\_free}
\end{itemize}

\noindent So for instance, the user can do:

\begin{verbatim}
  # ganglia swap_total cpu_speed cpu_num
\end{verbatim}

\noindent to produce the following output:

\begin{verbatim}
  startx          265064          851     1
  xcvs            1582384         733     2
\end{verbatim}

Also included in the extras directory is a small example script to add
some network statistics to the ganglia data stream.
This script was developed by Goneri Le Bouderxi
$<$\mailto{glebouder@mandrakesoft.com}$>$ of MandrakeSoft.
        
Goneri's \cmd{ganglia-network.sh} script can be run
periodically by the \cmd{cron} system to add network statistics to the
Ganglia data stream. This is just an example script that is not officially
part of the OSCAR cluster tool suite. No guarantees are included that this
script will work in current or future versions of Ganglia that may be
packaged as part of the OSCAR cluster tool suite.

