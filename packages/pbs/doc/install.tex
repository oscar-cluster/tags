% -*- latex -*-
%
% $Id: install.tex,v 1.6 2002/06/15 02:48:51 jenos Exp $
%
% $COPYRIGHT$
%

\subsection{The Portable Batch System (PBS) and Maui Scheduler}
\label{app:pbs-overview}

PBS serves as the job launcher and queuing system for OSCAR.  OSCAR
uses the open source version of PBS, OpenPBS.  A commercial version
(PBSPro) exists as well.  OpenPBS includes a basic FIFO scheduler with
it, which is disabled by default in OSCAR.  A more robust, open source
scheduler is enabled instead; called Maui.  In OSCAR 1.3, we are using
OpenPBS version 2.2p11 and Maui 3.06p4.  Both PBS and Maui will be
updated in future OSCAR releases.

Basic PBS functionality is tested by OSCAR's test suite, and is also
used to launch jobs when testing other software included in OSCAR.  If
the PBS test passes, PBS and Maui are up and working.  

If the users of your OSCAR cluster have not used PBS before, you can
expect somewhat of a learning curve.  The OSCAR user's documentation
contains some useful information to get them started.  The user's
document instructs users to ask the system administrator to provide
them with sample PBS scripts used in the OSCAR test suite.  These
scripts can be found in the home directory of the \user{oscartst} user
if the tests have been run previously.

\subsubsection{Configuring PBS}

By default, PBS installs without any queues or cluster specific
paremeters defined.  OSCAR configures PBS with sensible defaults based
on what it finds in the SIS database.  Whenever the ``Complete Cluster
Setup'' step is executed from the wizard, the \cmd{post\_install}
script from the PBS package in OSCAR is called.  The
\cmd{post\_install} configures only PBS parameters that are
non-existent, so as not to overwrite customizations by the system
administrator.  However, if you wish to force the default values back
in place over any customizations, the \cmd{post\_install} script can
be called manually with a \cmd{--default} option.  This will revert
all values to the original OSCAR settings.

\cmd{qmgr} can be used to configure queues and PBS server parameters.
The PBS \cmd{post\_install} uses \cmd{qmgr} behind the scenes.  There
are man pages available, but reading the PBS admin guide is the best
way to learn how to use it.  It is available on OpenPBS's homepage,
listed below.  You will have to create an account on their site in
order to download the admin guide.

\subsubsection{PBS Resources}

Abitrary node properties can be set by the administrator.  PBS calls
these properties ``resources''.  These resources can be specified on
the \cmd{qsub} command line when a user submits a job.  This allows a
user to restrict their jobs to run only on nodes exhibiting certain
properties.  If some nodes of a cluster have more memory, a different
network, faster prcoessors, etc., jobs can be submitted so they only
run a specific subset.  These properties are stored in plain text in
\file{/usr/spool/PBS/server\_priv/nodes}.  However, if adjusted in the
plain text file, the PBS server must be restarted in order for changes
to take effect.  The more elaborate method is to use the \cmd{qmgr}
command to modify node properties via the PBS API.  OSCAR gives each
node a starting property of {all}.

\subsubsection{An FAQueue}

A popular misconception about PBS queues is that they are bound to a
group of nodes.  This is false.  If you have a four node queue
defined, it is not associated with any specific nodes.  You can think
of a queue as a multidimensional box that a job must fit in in order
to allow submission.  i.e., the submitted parameters must fit within
certain max and min values for nodes, ppn (procs per node), walltime,
etc.  If specific nodes are desired to run on, then resource
attributes must be defined.

If you would like to get a full dump of your PBS server and queue
configuration, you can issue this command:
\cmd{qmgr -c "print server"}
The qmgr interface can be used to define additional queues and their
parameters.  You can also change the parameters on the default OSCAR
queue, ``workq''.  Be aware that if you call the \cmd{post\_install} 
command with the \cmd{--default} option, you will lose your customizations.
Also note that OSCAR's default wallclock limit on workq is 1 hour.  
Depending on your application, you may wish to adjust this.

Some useful links:

\begin{itemize}
\item OpenPBS: \url{http://www.openpbs.org/}

\item PBSPro: \url{http://www.pbspro.com/}

\item Maui Scheduler: \url{http://www.supercluster.org/}
\end{itemize}

% LocalWords:  Exp
