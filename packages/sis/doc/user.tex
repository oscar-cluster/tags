% -*- latex -*-
%
% $Id: user.tex,v 1.4 2002/06/18 19:03:22 mchasal Exp $
%
% $COPYRIGHT$
%

\subsection{An overview of SIS}
\label{app:sis-overview}

The System Installation Suite, or SIS, is a tool for installing 
linux systems over a network. It is used in OSCAR to install the
client nodes. SIS also provides the database from which OSCAR
obtains its cluster configuration information.

The main concept to understand about SIS is that it is an
\emph{image based} install tool. An image is basically a copy
of all the files that get installed on a client. This image
is stored on the server and can be accessed for customizations or
updates. You can even \cmd{chroot} into the image and perform builds.

Once this image is built, clients are defined and associated with the
image. When one of these clients boots using a SIS auto-install
environment, either on floppy, CD, or through a network boot, the
corresponding image is pulled over the network using \cmd{rsync}. 
Once the image is installed, it is customized with the hardware and
networking information for that specific client and it is then rebooted.
When booting the client will come up off the local disk and be ready
to join the OSCAR cluster.

\subsubsection{Building a SIS image}

Normally, an OSCAR image is built using the \button{Build OSCAR Client Image} 
button on the OSCAR wizard. This button brings up a panel that is actually 
directly from the SIS gui \cmd{tksis}. Once the information is filled in 
the SIS command \cmd{mksiimage} is invoked to actually build the image. 

In addition to building an image, you can use \cmd{tksis} or \cmd{mksiimage}
to delete images as well. Images can take a fair amount of disk space, so if
you end up with images that you aren't using, you can delete them to recover 
some space.

\subsubsection{Managing SIS clients}

Much like the OSCAR image creation, the \button{Define OSCAR Clients} button actually
invokes a \cmd{tksis} panel. There are a couple of SIS commands that are used 
to manage the client definitions. \cmd{mksirange} is used to define a group of
clients. More importantly \cmd{mksimachine} can be used to update client 
definitions. If, for example, you needed to change the MAC address after replacing
one of your clients, you could use \cmd{mksimachine}.

\subsubsection{Maintaining your client software}

There are many different ways to maintain the software installed on the client
nodes. Since SIS is image based, it allows you to also use an image based 
maintenance scheme. Basically, you apply updates and patches to your images 
and then resync the clients to their respective images. Since \cmd{rsync} is
used, only the actual data that has changed will be sent over the network to 
the client. The SIS command \cmd{updateclient} can be run on any client to
initiate this update.

\subsubsection{Additional information}

To obtain more detailed information about SIS, please refer to the many man
pages that are shipped with SIS. Some of the more popular pages are:

\begin{itemize}
\item tksis
\item mksiimage
\item mksidisk
\item mksirange
\item mksimachine
\item systemconfigurator
\item updateclient
\end{itemize}

You can also access the mailing lists and other docs through the sisuite
home page, \url{http://sisuite.org/}.
