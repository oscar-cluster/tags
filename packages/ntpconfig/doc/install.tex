% -*- latex -*-
%
% $Id: install.tex,v 1.4 2003/03/06 23:08:45 brechin Exp $
%
% $COPYRIGHT$
%

\subsection{Managing NTP for the OSCAR Server and Clients}
\label{app:ntp-overview}

NTP is the Network Time Protocol which is used to synchronize the
computer clock to external sources of time.  The \file{ntpd} daemon
can operate as a client (by connecting to other NTP servers to get the
current time) and as a server (by providing the current time to other
NTP clients).

The ntpconfig package which configures NTP for your OSCAR cluster,
has been completely re-written since OSCAR 4.1.

\subsubsection{Configuring ntpconfig}

If not configured, the package will set
up your cluster nodes such that they will use the headnode as the time 
server and sync time against it.  This mode is useful if you do not 
have an Internet connection.

The package will not overwrite any of your existing configurations for
NTP, e.g. during the initial OS installation on your
headnode, you set up NTP to sync time with \file{time.apple.com}, the
package will not change that and will simply set up your headnode as
the time server for your cluster nodes.  As a result, your compute
nodes' time will be synced relative to \file{time.apple.com}'s clock.

If you choose to configure the ntpconfig package, you can select a
NTP server to sync time with (we recommend \file{pool.ntp.org}), and your
headnode will then synchronize clock with the NTP server you chose and
any other time servers previously configured manually by modifying the
\file{/etc/ntp.conf} file.

If you have previously configured NTP using the ntpconfig package, 
and have configured it to use another server, the previous setting
will be overwritten.

\subsubsection{Enabling/Disabling the NTP Service}

By default, the \file{ntpd} daemon is configured to start at boot time
in run levels 2 through 5.  If for some reason you want to disable NTP
without actually uninstalling it, execute the following commands:

\begin{verbatim}
 # /etc/init.d/ntpd stop
 # /sbin/chkconfig --level 2345 ntpd off
\end{verbatim}

This will not only stop any currently running \file{ntpd} daemon, but
also prevent NTP from starting up at boot time.

\bigskip 

{\bf NOTE:} You must be {\tt root} to execute these commands.

\bigskip 

To restart NTP and make NTP start up at boot time, execute the following
commands:

\begin{verbatim}
 # /etc/init.d/ntpd restart
 # /sbin/chkconfig --level 2345 ntpd on
\end{verbatim}

For more information on NTP, see the (rather lengthy) documentation at
\url{http://www.ntp.org/}.
