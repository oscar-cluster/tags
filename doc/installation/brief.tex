% -*- latex -*-
%
% $Id: brief.tex,v 1.11 2002/06/11 20:52:57 mchasal Exp $
%
% $COPYRIGHT$
%

\section{Quick Start OSCAR Installation on a Private Subnet}

%%%%%%%%%%%%%%%%%%%%%%%%%%%%%%%%%%%%%%%%%%%%%%%%%%%%%%%%%%%%%%%%%%%%%%%%%%

\subsection{Why you shouldn't do a Quick Start install}

If you meet the following criteria and are very brave, you can 
try to install your cluster using the brief, mysterious,
cryptic, terse, and obscure documentation in this section.  
Otherwise, please do the right thing and read and use the 
detailed installation procedure section of this installation guide.
If you don't understand any of the following criteria, 
then you probably don't meet them:

\begin{enumerate}
\item You hate to read manuals, documentation, or other printed matter.
\item You are a unix guru. (You can uset \cmd{netcfg} or an editor to
  configure network interfaces.)
\item Your cluster client machines are on a private subnet.
\item Your cluster server machine has two network interfaces, one public,
  and one connected to the cluster client machines.
\item The default RPM list supplied by OSCAR for your distribution
  (\file{oscar-\oscarversion/oscarsamples/<distro>-<version>-<arch>.rpmlist}) is
  acceptable for your clients.
\item You are using a distribution that is listed as \emph{Fully supported} 
 in Table~\ref{tab:oscar-distro-support}.
\item You know how to install your chosen distribution on a machine.
\item You've never read completely through the installation instructions
  of anything, ever. 
\end{enumerate}

If you decide that the quick install isn't for you, see the detailed installation
instructions in section \ref{det:top}.Throughout the quick install instructions,
there is a reference to the appropriate section in the detailed installation 
instructions in parenthesis at the end of each step.

\subsection{Quick installation procedures}

Note: All actions specified herein should be performed by the
\user{root} user on the service machine unless noted otherwise.

\begin {enumerate}

\item Install (or already have installed) Linux on your server machine.
The only requirements for your Linux installation are:

        \begin{enumerate}
        \item There should be approximately 2 gigabytes free space in both
        the \file{/} and \file{/var} filesystems. It is simplest if you
        just create 1 large partition on the entire drive.(Detail:Section
        \ref{det:serverdiskpar})
        \item Some X environment such as GNOME or KDE must be installed.
        \item Networking must be set up and working on the public interface.
        (Do yourself a favor and install some type of network security if
        your system is exposed to the general internet.)
        \item The second network interface for the private cluster network
        must be installed.
        \end{enumerate}
(Detail:Section \ref{det:serverosinstall})

\item After the installation is complete, log on as root to the server 
machine.

\item Download and unpack OSCAR with these commands:

% We have to use \tt instead of {verbatim} because we need to use
% \oscarversion inside.  This makes it somewhat painful -- much less
% easy than {verbatim}.
\vspace{11pt}
{\tt
  \# cd $^\sim$ \\
  \# ncftp ftp.sourceforge.net \\
  ncftp / > cd pub/sourceforge/oscar \\
  ncftp /pub/sourceforge/oscar > get oscar-\oscarversion.tar.gz \\
  ncftp /pub/sourceforge/oscar > quit \\
  \# tar -zxf oscar-\oscarversion.tar.gz
}
\vspace{11pt}

Ignore the configure script, it is not used in the installation process
of OSCAR.

(Detail:Section \ref{det:unpack})

\item Copy the rpms from \emph{all} of the distribution CD's to \file{/tftpboot/rpm}
using these, or similar, commands:

\begin{verbatim}
  (insert cd)
  # cp /mnt/cdrom/Redhat/RPMS/*.rpm /tftpboot/rpm
\end{verbatim}
(Detail:Section \ref{det:rpmcopy})

\item Configure the second (private) cluster network adapter using the linux
\cmd{/usr/sbin/netcfg} command or your favorite editor. Set the interface 
IP address to 10.0.0.250, set the interface configuration protocol
to ``none'', and set the interface to activate at boot time.
Then reboot your machine. and make sure that the private cluster
interface is properly setup and activated. 
(Detail:Section \ref{det:serveradapter})

\item After the reboot is complete, log on as root to the server 
machine.

\item To start the OSCAR cluster installation,  in the X environment do
the following command:

% We have to use \tt instead of {verbatim} because we need to use
% \oscarversion inside.  This makes it somewhat painful -- much less
% easy than {verbatim}.
\vspace{11pt}
{\tt
  cd $^\sim$/oscar-\oscarversion \\
  ./install\_cluster eth1
}
\vspace{11pt}
  
In the above command, substitute the device name 
(e.g., \emph{eth1})
for your server's private network ethernet adapter. After \cmd{install\_cluster}
successfully completes some configuration, it will display 
the OSCAR wizard.
(Detail:Section \ref{det:installcluster})

\item Press the \button{Build OSCAR Client Image} button. 
This pops up the "Create a SystemImager Image" window. Verify that the last
part of the filename in the \field{Disk Partition File} field matches
the type of drives that are in your clients. OSCAR ships \file{sample.disk.ide}
\& \file{sample.disk.scsi} in the \file{oscarsamples} directory.
Press the \button{BuildImage} button to build a SystemImager image for installation to
the compute nodes.  You will see a progress bar across the bottom
of the window as the image is built.  A dialog will pop up when the build is complete. 
If it is successful, press the \button{Close} button on the pop up window 
and then the \button{Close} button on the "Build a SystemImager
Image" window. You have just built an installation image on the
server, with the name \file{oscarimage}.
(Detail:Section \ref{det:buildimage})

\item Press the \button{Define OSCAR Clients} button on the OSCAR Wizard. 
On the window that is opened, fill in the number of client nodes that your cluster
contains. Verify that the IP information is correct and correct if it is not.
Pay special attention to the domain name. Make sure one is filled in as it is
required.
Press the \button{Addclient} button to define the clients. Once that is 
complete a dialog will pop up. 
If it is successful, press the \button{Close} button on the pop up window 
and then the \button{Close} button on the "Define OSCAR Clients"
window. 
(Detail:Section \ref{det:defclients})

\item Press the \button{Setup Networking} button on the OSCAR Wizard.
The window that is opened will help you collect MAC addresses and configure
the remote boot services. If your client nodes do not support PXE booting,
you will need to create a boot diskette. Put a diskette in the floppy
drive and press the \button{Build Autoinstall Floppy} button. You may create 
several diskettes if you like.
(Detail:Section \ref{det:setupnetwork})

\item Press the \button{Collect MAC Addresses} button. The wizard will start
scanning the network as indicated in the message at the top of the window. 
(Detail:Section \ref{det:setupnetwork})

\item Now, you need to network boot your nodes. This is done in one of 2 ways:

        \begin{itemize}
        \item If your nodes support PXE booting, set the boot order in the BIOS
        to have the network adapter first.
        \item If your nodes do not support PXE, insert the boot floppy created
        in the previous step and boot off the floppy
        \end{itemize}

(Detail:Appendix \ref{app:net-boot-client-nodes})

\item As the nodes broadcast over the network, the MAC addresses detected will show
up on the left side of the window. Select a MAC address and select its 
corresponding client. Press the \button{Assign Mac to Node} button to give the
node that MAC address. 
(Detail:Section \ref{det:setupnetwork})

\item When you have assigned a MAC address to each node, press the \button{Stop Collecting}
button to stop scanning the network. Then press the \button{Configure DHCP Server}
to setup the server to answer the client requests.
(Detail:Section \ref{det:setupnetwork})

\item If your clients support PXE booting, press the \button{Setup Network Boot} 
button to configure the server to answer PXE requests.
(Detail:Section \ref{det:setupnetwork})

\item Press \button{Close} to dismiss the network setup window.

\item Network boot the clients, as you did when collecting the MAC addresses,
again. This time the clients will be answered by the server and will perform
the installation. The install should only take about 3-5 minutes depending on 
your hardware and your network. You can install multiple clients simultaneously.
(Detail:Section \ref{det:clientinstall})

\item As each client finishes, it will start beeping and printing a message to the 
console. You should now reboot them off of thier newly installed hard disks.

        \begin{itemize}
        \item If you are using PXE boot, reboot the clients and set the BIOS
        boot order to boot to hard disk before the network.
        \item If you are using auto install diskettes, just remove the diskette
        and reboot
        \end{itemize}
(Detail:Section \ref{det:clientfinish})

\item To verify that the clients are installed properly use
  \cmd{ping\_clients} script.  In a terminal as the root user issue
  \cmd{$^\sim$/oscar-\oscarversion/scripts/ping\_clients}.
  (Detail:Section \ref{det:pingclients})

\item Once all the clients have successfully booted, Press the 
\button{Complete Cluster Setup} button. Your OSCAR cluster is
now installed and configured.
(Detail:Section \ref{det:completeinstall})

\item To verify proper installation, you should test the cluster.
OSCAR provides a test suite to verify the basic cluster
functions. To perform the test, do the following:

        \begin{enumerate}
        \item Press the \button{Test Cluster Setup} button. A window will
        pop up that will ask for a non-root user. 
        \item Enter the user name of a non-root user. If this user doesn't
        exist it will create it. The OSCAR test suite will be installed
        in the user's home directory.
                \begin{enumerate}
                \item Open a new terminal and login as the user given above.
                \item change directory to \file{$^\sim$/OSCAR\_test}
                \item run \cmd{./test\_cluster}
                \item enter the number of cluster nodes and the number of 
                processors per node when prompted. The test will submit
                simple jobs to the cluster and print the output. If there
                are any errors, there is a problem in your installation.
                \end{enumerate}
         \end{enumerate}
(Detail:Section \ref{det:testcluster})

\end{enumerate}
