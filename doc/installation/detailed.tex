% -*- latex -*-
%
% $Id: detailed.tex,v 1.29 2002/07/16 14:39:46 jsquyres Exp $
%
% $COPYRIGHT$
%

\section{Detailed Cluster Installation Procedure}
\label{sec:detail}

All actions specified herein should be performed by the \user{root}
user on the server node unless noted otherwise.  Note that if you
login as a regular user and use the \cmd{su} command to change to the
\user{root} user, you {\em must} use ``\cmd{su -}'' to get the full
\user{root} environment.  Using ``\cmd{su}'' (with no arguments) may
not be sufficient, and may cause obscure errors during an OSCAR
installation.

%%%%%%%%%%%%%%%%%%%%%%%%%%%%%%%%%%%%%%%%%%%%%%%%%%%%%%%%%%%%%%%%%%%%%%%%%%
%%%%%%%%%%%%%%%%%%%%%%%%%%%%%%%%%%%%%%%%%%%%%%%%%%%%%%%%%%%%%%%%%%%%%%%%%%

\subsection{Server Installation and Configuration}
\label{det:serverinstall}
  
During this phase, you will prepare the machine to be used as the
server node in the OSCAR cluster.

%%%%%%%%%%%%%%%%%%%%%%%%%%%%%%%%%%%%%%%%%%%%%%%%%%%%%%%%%%%%%%%%%%%%%%%%%%

\subsubsection{Install Linux on the server machine} 
\label{det:serverosinstall}

If you have a machine you want to use that already has Linux
installed, ensure that it meets the minimum requirements as listed in
Section~\ref{sec:intro-min-sys}.  If it does, you may skip ahead to
Section~\ref{det:serverdiskpar}.

It should be noted that OSCAR is only supported on the distributions
listed in Table~\ref{tab:oscar-distro-support}
(page~\pageref{tab:oscar-distro-support}).  As such, use of
distributions other than those listed will likely require some porting
of OSCAR, as many of the scripts and software within OSCAR are
dependent on those distributions. 

When installing Linux, do not worry about doing a ``custom'' install
since OSCAR will install all the software on which it depends.  The
only other Linux installation requirement is that some X windowing
environment such as GNOME or KDE must be installed.  Therefore, a
typical ``workstation'' install is usually sufficient.

\begin{discuss}
  Is this paragraph still relevant?
\end{discuss}

If you install RedHat on the server node, during the installation you
should enable the ipchains-base firewall that is included with the
RedHat distribution in medium mode.  Other firewalls that are stronger
and more versatile can be installed later, but this will offer some
protection until that time.  Note that OSCAR currently assumes that
only the server node is exposed to the general network, with the
server and the rest of the cluster's nodes being on a private network.
To keep the RedHat firewall from interfering with network traffic
between the server node and the other nodes in the cluster, OSCAR
automatically disables portions of the RedHat firewall.  This may not
have the intended results in the situation where all the cluster nodes
are exposed on the general network.  See Appendix~\ref{app:security}
for more information about firewalls and other security software that
can be installed.

%%%%%%%%%%%%%%%%%%%%%%%%%%%%%%%%%%%%%%%%%%%%%%%%%%%%%%%%%%%%%%%%%%%%%%%%%%

\subsubsection{Disk space and directory considerations}
\label{det:serverdiskpar}

OSCAR has certain requirements for server disk space. Space will be
needed to store the Linux RPMs and to store the images.  The RPMs will
be stored in \file{/tftpboot/rpm}. Approximately 1GB is required to
store the RPMs.  The images are stored in \file{/var/lib/systemimager}
and will need approximately 1GB per image. Only one image is required,
although you may want to create more in the future.

If you are installing a new server, it is suggested that you allow for
2GB in both the \file{/} (which contains \file{/tftpboot}) and
\file{/var} filesystems when partitioning the disk on your server.

If you are using an existing server, you will need to verify that you
have enough space on the disk partitions. Again 2GB of free space is
recommended in both the \file{/} and \file{/var} partitions.

You can check the amount of free space on your drive's partitions by
issuing the command \cmd{df -h} in a terminal.  The result for each
file system is located below the \cmd{Avail} column heading. If your
root (\file{/}) partition has enough free space, enter the following
command in a terminal:

\begin{verbatim}
  # mkdir -p /tftpboot/rpm
\end{verbatim}
  
If your root partition does not have enough free space, create the
directories on a different partition that does have enough free space
and create links to them from the root (\file{/}) directory.  For
example, if the partition containing \file{/usr} contains enough
space, you could do so by using the following commands:

\begin{verbatim}
  # mkdir -p /usr/tftpboot/rpm
  # ln -s /usr/tftpboot /tftpboot
\end{verbatim}

The same procedure should be repeated for the
\file{/var/lib/systemimager} subdirectory.

%%%%%%%%%%%%%%%%%%%%%%%%%%%%%%%%%%%%%%%%%%%%%%%%%%%%%%%%%%%%%%%%%%%%%%%%%%
    
\subsubsection{Get a copy of OSCAR and unpack on the server} 
\label{det:unpack}

If you are reading this, you probably already have a copy. If not, go
to \url{http://oscar.sourceforge.net/} and download the latest OSCAR
Regular or Extra Crispy distribution package (see
Section~\ref{sec:download}, page~\pageref{sec:download}).

\begin{discuss}
  Is \file{/usr/local/oscar} really still reserved?  Or is it
  \file{/opt/oscar} nowadays?
\end{discuss}

Copy the OSCAR distribution package to a directory such as
\user{root}'s home directory on the server node.  Although there is no
required installation directory (note that you may not use
\file{/usr/local/oscar} -- it is reserved for special use), the rest
of these instructions will assume that you downloaded the OSCAR
distribution package to \user{root}'s home directory.

Do {\bf not} unpack the tarball on a Windows based machine and copy
the directories over to the server, as this will convert all the
scripts to ``DOS'' format and will render them useless under Linux.

Open a command terminal and issue the following commands to unpack the
OSCAR distribution package (Regular):

\begin{verbatim}
  # cd ~
  # tar zxf <filename>
\end{verbatim}

Where \file{$<$filename$>$} is either
\file{oscar-\oscarversion.tar.gz} (regular distribution) or
\file{oscar\--including\--srpms\--\oscarversion.tar.gz} (extra crispy
distribution).

\def\obase{$^\sim$/oscar-\oscarversion}

\begin{table}[htbp]
  \begin{center}
    \begin{tabular}{|l|p{3in}|}
      \hline
      \multicolumn{1}{|c|}{Directory} &
      \multicolumn{1}{|c|}{Contents} \\
      \hline
      \hline
      \file{\obase/} & the base OSCAR directory \\
%
      \file{\obase/COPYING} & GNU General Public License
      v2 \\
%
      \file{\obase/doc} & OSCAR documentation directory \\
%
      \file{\obase/images} & auxiliary images used in the GUI \\
%
      \file{\obase/install\_cluster} & main installation script \\
%
      \file{\obase/lib} & auxiliary library routines \\
%
      \file{\obase/oscarsamples} & sample configuration files \\
%
      \file{\obase/packages} & RPM and installation files for the
      OSCAR packages \\
%
      \file{\obase/README} & text README document \\
%
      \file{\obase/scripts} & contains scripts that do most of the
      work \\
%
      \file{\obase/share} & more auxiliary helper files \\
%
      \file{\obase/testing} & contains OSCAR cluster test software \\
      \hline
    \end{tabular}
    \caption{OSCAR distribution package file and directory layout.}
    \label{tab:oscar-dir-struct}
  \end{center}
\end{table}

%%%%%%%%%%%%%%%%%%%%%%%%%%%%%%%%%%%%%%%%%%%%%%%%%%%%%%%%%%%%%%%%%%%%%%%%%%

\subsubsection{Configure the ethernet adapter for the cluster} 
\label{det:serveradapter}

Assuming you want your server to be connected to both an external
network and the internal cluster subnet, you will need to have two
ethernet adapters installed in the server. It is preferred that you do
this because exposing your cluster may be a security risk, and certain
software used in OSCAR (such as DHCP) may conflict with your external
network.  

Once both adapters have been physically installed, you need to
configure them.\footnote{Beware of 3COM network cards; there are
  certain models which simply do not work under Linux.  See the OSCAR
  web site for more information.}  The following section describes the
use of the \cmd{neat} network configurator; however, any network
configurator (including a text editor, if you know which files to
edit) will do.

Once you have booted Linux into an X environment, open a terminal and
enter the command:

\begin{verbatim}
  # neat &
\end{verbatim}
  
The network configuration utility will be started, which you will use
to configure your network adapters.
  
At this point, the \panel{Names} panel will be active. On this panel
you will find the settings for the server's hostname, domain,
additional search domains, and name servers. All of this information
should have been filled in by the standard Linux installation. 

To configure your ethernet adapters, you will need to first press the
\button{Interfaces} button to bring up the panel that allows you to
update the configuration of all of your server machines interfaces.
You should now select the interface that is connected to the cluster
network by clicking on the appropriate device. If your external
adapter is configured on device ``\file{eth0}'', then you should most
likely select ``\file{eth1}'' as the device, assuming you have no
other adapters installed. 

After selecting the appropriate interface, press the \button{Edit}
button to update the information for the cluster network adapter.
Enter a private IP address\footnote { There are three private IP
  address ranges: 10.0.0.0 to 10.255.255.255; 172.16.0.0 to
  172.32.255.255; and 192.168.0.0 to 192.168.255.255.  Additional
  information on private intranets is available in RFC 1918.  You
  should not use the IP addresses 10.0.0.0 or 172.16.0.0 or
  192.168.0.0 for the server.  If you use one of these addresses, the
  network installs of the client nodes will fail.}  and the associated
netmask\footnote{The netmask 255.255.255.0 should be sufficient for
  most OSCAR clusters.}  in their respective fields. Additionally, you
should be sure to press the \button{Activate interface at boot time}
button and set the \button{Interface configuration protocol} to
``none''.

After completing the updates, press the \button{Done} button to return
to the main utility window, pressing the \button{Save} button in the
Save current configuration menu that pops up.  Then press the
\button{Save} button at the bottom of the main network configuration
window to confirm your changes, and then press the \button{Quit} to
leave the network configuration utility.
  
Now reboot the server node to ensure that all the changes are
propagated to the appropriate configuration files. To confirm that all
ethernet adapters are in the ``up'' state, once the machine has
rebooted, open another terminal window and enter the following
command:

\begin{verbatim}
  # ifconfig -a
\end{verbatim}
  
You should see {\tt UP} as the first word on the third line of output
for each adapter. If not, there is a problem that you need to resolve
before continuing. Typically, the problem is that the wrong module is
specified for the given device. Try using the network configuration
utility again to resolve the problem.
  
%%%%%%%%%%%%%%%%%%%%%%%%%%%%%%%%%%%%%%%%%%%%%%%%%%%%%%%%%%%%%%%%%%%%%%%%%%

\subsubsection{Copy distribution installation RPMs to \file{/tftpboot/rpm}}
\label{det:rpmcopy}

In this step, you need to copy the RPMs included with your Linux
distribution into the \file{/tftpboot/rpm} directory.  When each CD is
inserted, Linux should automatically make the contents of the CD be
available in the \file{/mnt/cdrom} directory.  Then for each CD,
locate the directory that contains the RPMs.  In RedHat 7.1, the RPMs
are located in the \file{RedHat/RPMS} directory (i.e.,
\file{/mnt/cdrom/RedHat/RPMS}).  After locating the RPMs on the each
CD, copy them into \file{/tftpboot/rpm} with a command such as:

\begin{verbatim}
  # cp /mnt/cdrom/RedHat/RPMS/*.rpm /tftpboot/rpm
\end{verbatim}
  
Be sure to repeat the above process for all CDs.  After using each CD
you will have to unmount it from the local file system and eject it by
issuing these commands:

\begin{verbatim}
  # cd ~
  # eject cdrom
\end{verbatim}

%%%%%%%%%%%%%%%%%%%%%%%%%%%%%%%%%%%%%%%%%%%%%%%%%%%%%%%%%%%%%%%%%%%%%%%%%%

\subsubsection{Copy distribution update RPMs to \file{/tftpboot/rpm}}
\label{det:distro-updates}

Obtain all the relevant RPM updates for your distribution -- security
patches, bug fixes, etc.  Most distributions (including RedHat)
include a tool to automate this process (e.g., \cmd{up2date}) that
will automatically download and install all relevant updated RPMs.

Ensure that these updated RPMs are also copied to
\file{/tftpboot/rpm}.  It does not matter if two different versions of
the same RPM end up in \file{/tftpboot/rpm} (e.g.,
\file{foo-1.2.3-1.i386.rpm} from the installation CD and
\file{foo-1.2.4-1.i386.rpm} from the updates) -- OSCAR will
automatically use the latest version when it selects RPMs to install.

%%%%%%%%%%%%%%%%%%%%%%%%%%%%%%%%%%%%%%%%%%%%%%%%%%%%%%%%%%%%%%%%%%%%%%%%%%
%%%%%%%%%%%%%%%%%%%%%%%%%%%%%%%%%%%%%%%%%%%%%%%%%%%%%%%%%%%%%%%%%%%%%%%%%%
  
\subsection{Initial OSCAR Server Configuration}

During this phase, the software needed to run OSCAR will be installed
on the server. In addition, some initial server configuration will be
performed. 

The following steps must be run within an X windowing environment.

%%%%%%%%%%%%%%%%%%%%%%%%%%%%%%%%%%%%%%%%%%%%%%%%%%%%%%%%%%%%%%%%%%%%%%%%%%

\subsubsection{Change to the OSCAR directory and run \file{install\_cluster}}
\label{det:installcluster}

Change directory to the top-level OSCAR directory and start the OSCAR
install wizard:

% We have to use \tt instead of {verbatim} because we need to use
% \oscarversion inside.  This makes it somewhat painful -- much less
% easy than {verbatim}.
\vspace{11pt}
{\tt
\# cd /root/oscar-\oscarversion \\
\indent \# ./install\_cluster $<$device$>$
}
\vspace{11pt}
  
In the above command, substitute the device name (e.g., \emph{eth1})
in place of \cmd{$<$device$>$} for your server's private network
ethernet adapter.  The script will first run the part one server
configuration script, which does the following:

\begin{enumerate}
\item copies OSCAR RPMs to \file{/tftpboot/rpm}
\item installs all OSCAR server RPMs
\item updates \file{/etc/hosts} with OSCAR aliases
\item updates \file{/etc/exports} 
\item adds OSCAR paths to \file{/etc/profile} 

  \begin{discuss}
    Do we really change /etc/profile?
  \end{discuss}

\item updates system startup (\file{/etc/rc.d/init.d}) scripts
\item restarts affected services
\item if all the above is successful, launch the OSCAR GUI
  installation wizard
\end{enumerate}
  
The wizard, as shown in Figure~\ref{fig:detailed-oscar-wizard}, is
provided to guide you through the rest of the cluster installation.
To use the wizard, you will complete a series of steps, with each step
being initiated by the pressing of a button on the wizard. Do not go
on to the next step until the instructions say to do so, as there are
times when you must complete an action outside of the wizard before
continuing on with the next step. For each step, there is also a
\button{Help} button located directly to the right of the step button.
When pressed, the \button{Help} button displays a message box
describing the purpose of the step.

\begin{figure}[htbp]
  \begin{center}
    \includegraphics[scale=\imgscale]{figs/2_sbs-oscar-wizard.\figext}
    \caption{OSCAR Wizard.}
    \label{fig:detailed-oscar-wizard}
  \end{center}
\end{figure}
  
As each of the steps are performed, there is output generated that is
displayed to the user. 

%%%%%%%%%%%%%%%%%%%%%%%%%%%%%%%%%%%%%%%%%%%%%%%%%%%%%%%%%%%%%%%%%%%%%%%%%%
%%%%%%%%%%%%%%%%%%%%%%%%%%%%%%%%%%%%%%%%%%%%%%%%%%%%%%%%%%%%%%%%%%%%%%%%%%

\subsection{Cluster Definition}
\label{sec:detailed-cluster-def}

During this phase, you will complete steps 1 through 6 of the OSCAR
wizard in defining your cluster.  If you encounter problems or wish to
redo any of the SIS actions performed in the wizard steps 2 or 3,
please refer to the \cmd{SIS(1)} man pages.

%%%%%%%%%%%%%%%%%%%%%%%%%%%%%%%%%%%%%%%%%%%%%%%%%%%%%%%%%%%%%%%%%%%%%%%%%%

\subsubsection{Step 1: Prepare OSCAR server for install} 
\label{det:prepareforinstall}

Begin the installation process by pressing the Step 1 button of the
wizard entitled \button{Prepare OSCAR Server for Install}. 

Here you will select the default MPI implementation for use on the
cluster.  Note that all available MPI implementations will be
available for general use; this step simply picks which one will be
the system default.

This step also installs and starts other necessary software, devices,
and services.

%%%%%%%%%%%%%%%%%%%%%%%%%%%%%%%%%%%%%%%%%%%%%%%%%%%%%%%%%%%%%%%%%%%%%%%%%%

\subsubsection{Step 2: Build the Image} 
\label{det:buildimage}

Press the Step 2 button of the wizard entitled \button{Build OSCAR
  Client Image}. A dialog will be displayed. In most cases, the
defaults will be sufficient. You should verify that the disk partition
file is the proper type for your client nodes. The sample files have
the disk type as the last part of the filename. You may also want to
change the post installation action and the IP assignment methods.
\msg{It is important to note that if you wish to use automatic reboot,
  you should make sure the BIOS on each client is set to boot from the
  local hard drive before attempting a network boot by default. If you
  have to change the boot order to do a network boot before a disk
  boot to install your client machines, you should not use automatic
  reboot.}  Once you are satisfied with the input, click the
\button{Build Image} button.

Building the image will take a few minutes; the progress bar on the
bottom will give you the status and a small dialog will appear when
the image is complete.  If building the image fails, see Section
\ref{subsec:rh72notes}.
  
A sample dialog is shown in Figure~\ref{fig:detailed-build-image}.

\begin{figure}[htbp]
  \begin{center}
    \includegraphics[scale=\imgscale]{figs/4a_sbs-build-image1.\figext}
    \caption{Build the image.}
    \label{fig:detailed-build-image}
  \end{center}
\end{figure}
  
\paragraph{Customizing your image.}

The defaults of this panel use the sample disk partition and RPM
package files that can be found in the \file{oscarsamples} directory.
You may want to customize these files to make the image suit your
particular requirements.

\subparagraph{Disk partitioning.}

The disk partition file contains a line for each partition desired,
where each line is in the following format:

\begin{verbatim}
  <partition> <size in megabytes> <type> <mount point> <options>
\end{verbatim}

Here is a sample (for a SCSI disk):

\begin{verbatim}
  /dev/sda1       24          ext2      /boot   rw
  /dev/sda5       128         swap
  /dev/sda6       1000        ext2      /       rw
\end{verbatim}

\begin{discuss}
  The \file{sample.dist.scsi} actually has ``{\tt defaults}'' instead
  of ``{\tt rw}''.  Should that be listed here?
\end{discuss}

\begin{discuss}
  Should {\tt nfs\_oscar:/home - nfs /home rw} be listed here as well?
  It's in all three sample files in \file{oscarsamples}.
\end{discuss}

The last partition specified will grow to fill the entire disk.  You
can create your own partition files, but make sure that you do not
exceed the physical capacity of your client hardware. The sample
listed above, and some others, are in the \file{oscarsamples}
directory.

\subparagraph{Package lists.}

The package list is simply a list of RPM file names (one per line). Be
sure to include all prerequisites that any packages you might add.
You do not need to specify the version, architecture, or extension of
the RPM filename.  For example, \file{bash-2.05-8.i386.rpm} need only
be listed as ``\file{bash}''.

\subparagraph{Custom kernels.}

If you want to use a customized kernel, you can add it to the image
after it is built. Follow these steps:

\begin{enumerate}
\item Copy the kernel and associated files (\file{System.map},
  \file{module-info}, etc.) into the directory:

  \vspace{11pt}
  \centerline{\file{/var/lib/systemimager/images/<imagename>/boot}}

\item Edit the \file{systemconfig.conf} file in the following
  directory:

  \vspace{11pt}
  \centerline{\file{/var/lib/systemimager/images/<imagename>/etc/systemconfig/}}
  \vspace{11pt}
  
  Change the \emph{PATH} parameter in the \emph{[KERNEL0]} stanza to
  match your new kernel's filename.

\item When the clients are installed, they will copy over and boot
  your new kernel.
\end{enumerate}

Note that there is no \file{lilo.conf} file in the image. That file is
created by System Configurator from the contents of the
\file{systemconfigurator.conf} file. See the \file{systemconfig.conf}
man page for full details on the contents of this file.

%%%%%%%%%%%%%%%%%%%%%%%%%%%%%%%%%%%%%%%%%%%%%%%%%%%%%%%%%%%%%%%%%%%%%%%%%%

\subsubsection{Step 3: Define your client machines} 
\label{det:defclients}

Press the Step 3 button of the wizard entitled \button{Define OSCAR
  Clients}. In the dialog box that is displayed, enter the appropriate
information. Again the defaults will be correct in most cases. At a
minimum, you will need to enter a value in the \field{Number of Hosts}
to specifies how many clients you want to create.

\begin{enumerate}
  
\item The \field{Image Name} field should specify the image name that
  was used to create the image in Step 2.
  
\item The \field{Domain Name} field should be used to specify the
  client's IP domain name.  {\bf This field \emph{must} have a value.}
  It should contain the server node's domain (if it has one); if the
  server does not have a domain name, use any name, such as
  \emph{oscardomain}.

\item The \field{Base name} field is used to specify the first part of
  the client name and hostname. It will have an index appended to the
  end of it.

\item The \field{Number of Hosts} field specifys how many clients to
  create.
  
\item The \field{Starting Number} specifies the index to append to the
  \field{Base Name} to derive the first client name. It will be
  incremented for each subsequent client.

\item The \field{Starting IP} specifies the IP address of the first
  client. It will be incremented for each subsequent client.
  
\item The \field{Subnet Mask} specifies the IP netmask for all
  clients.
  
\item The \field{Default Gateway} specifies the default route for all
  clients.

\end{enumerate}
  
When finished entering information, press the \button{Addclients} button.
A sample dialog is shown in Figure~\ref{fig:detailed-define-clients}. 

After the clients are created, a dialog will pop up with the
completion status. After closing that, you may press the
\button{Close} button and continue with the next step.

\begin{figure}[htbp]
  \begin{center}
    \includegraphics[scale=\imgscale]{figs/5a_sbs-define-clients1.\figext}
    \caption{Define the Clients.}
    \label{fig:detailed-define-clients}
  \end{center}
\end{figure}
    
%%%%%%%%%%%%%%%%%%%%%%%%%%%%%%%%%%%%%%%%%%%%%%%%%%%%%%%%%%%%%%%%%%%%%%%%%%

\subsubsection{Step 4: Collect client MAC addresses and Setup Networking} 
\label{det:setupnetwork}

The MAC address of a client is a twelve hex-digit hardware address
embedded in the client's ethernet adapter. MAC addresses look like,
``{\tt 00:0A:CC:01:02:03}'', as opposed to the familiar format of IP
addresses. These MAC addresses uniquely identify client machines on a
network before they are assigned IP addresses. DHCP uses the MAC
address to assign IP addresses to the clients.

In order to collect the MAC addresses, press the Step 4 button of the
wizard entitled \button{Setup Networking}. The OSCAR network utility
dialog box will be displayed.  To use this tool, you will need to know
how to network boot your client nodes.  For instructions on doing so,
see Appendix~\ref{app:net-boot-client-nodes}. A sample dialog is shown
in Figure~\ref{fig:detailed-collect-mac}.

To start the collection, press the \button{Collect MAC Address} button
and then network boot the first client.  As the clients boot up, their
MAC addresses will show up in the left hand window. Select a MAC
address and the appropriate client in the right side window. Click
\button{Assign MAC to Node} to associate that MAC address with that
node. If you would like to make specific nodes associated with
specific client definitions, you should boot them one at a time. If
you do not care which node gets associated with which client, you may
boot them all at once and randomly assign the MAC addresses.

When you have collected all of the MAC addresses, click the
\button{Stop Collecting} button and then click the \button{Configure
  DHCP Server} button to configure the DHCP server.
 
You may also configure your remote boot method from this panel. The
\button{Build Autoinstall Floppy} button will build a boot floppy for
client nodes that do not support PXE booting. The \button{Setup
  Network Boot} button will configure the server to answer PXE boot
requests if your client hardware supports it. See
Appendix~\ref{app:net-boot-client-nodes} for more details.  

When you have collected the addresses for all your client nodes, and
completed the networks setup press \button{Close}.

\begin{figure}[htbp]
  \begin{center}
    \includegraphics[scale=\imgscale]{figs/6e_sbs-found-mac.\figext}
    \caption{Collect client MAC addresses.}
    \label{fig:detailed-collect-mac}
  \end{center}
\end{figure}

%%%%%%%%%%%%%%%%%%%%%%%%%%%%%%%%%%%%%%%%%%%%%%%%%%%%%%%%%%%%%%%%%%%%%%%%%%
%%%%%%%%%%%%%%%%%%%%%%%%%%%%%%%%%%%%%%%%%%%%%%%%%%%%%%%%%%%%%%%%%%%%%%%%%%

\subsection{Client Installations}
\label{det:clientinstall}

During this phase, you will network boot your client nodes and they
will automatically be installed and configured as specified in
Section~\ref{sec:detailed-cluster-def} above. For a detailed
explanation of what happens during client installation, see
Appendix~\ref{app:client-install}.

%%%%%%%%%%%%%%%%%%%%%%%%%%%%%%%%%%%%%%%%%%%%%%%%%%%%%%%%%%%%%%%%%%%%%%%%%%

\subsubsection{Network boot the client nodes}

See Appendix~\ref{app:net-boot-client-nodes} for instructions on
network booting clients.

%%%%%%%%%%%%%%%%%%%%%%%%%%%%%%%%%%%%%%%%%%%%%%%%%%%%%%%%%%%%%%%%%%%%%%%%%%

\subsubsection{Check completion status of nodes}
\label{det:clientfinish}

After a few minutes, the clients should complete the installation.
You can watch the client consoles to monitor the progress. Depending
on the Post Installation Action you selected when building the image,
the clients will either halt, reboot, or beep incessantly when the
installation is completed.

The time required for installation depends on the capabilities of your
server, your clients, your network, and the number of simultaneous
client installations.  Generally, it should complete within a few
minutes.
  
%%%%%%%%%%%%%%%%%%%%%%%%%%%%%%%%%%%%%%%%%%%%%%%%%%%%%%%%%%%%%%%%%%%%%%%%%%

\subsubsection{Reboot the client nodes}

After confirming that a client has completed its installation, you
should reboot the node from its hard drive. If you chose to have your
clients reboot after installation, they will do this on their
own. If the clients are not set to reboot, you must manually
reboot them. The filesystems will have been unmounted so it is safe
to simply reset or power cycle them.

\msg{Note: If you had to change the BIOS boot order on the client to
  do a network boot before booting from the local disk, you will need
  to reset the order to prevent the node from trying to do another
  network install.}

%%%%%%%%%%%%%%%%%%%%%%%%%%%%%%%%%%%%%%%%%%%%%%%%%%%%%%%%%%%%%%%%%%%%%%%%%%

\subsubsection{Check network connectivity to client nodes}
\label{det:pingclients}

In order to perform the final cluster configuration, the server must
be able to communicate with the client nodes over the network. If a
client's ethernet adapter is not properly configured upon boot,
however, the server will not be able to communicate with the client. A
quick and easy way to confirm network connectivity is to run the
following from the top-level OSCAR directory:

\begin{verbatim}
  # ./scripts/ping_clients
\end{verbatim}

The above commands will run the \cmd{ping\_clients} script, which will
attempt to ping each defined client and will print a message stating
success or failure. If a client cannot be pinged, there was a problem
configuring the ethernet adapter, and you will have to log in to the
machine and manually configure the adapter. Once all the clients have
been installed, rebooted, and their network connections have been
confirmed, you may proceed with the next step.

%%%%%%%%%%%%%%%%%%%%%%%%%%%%%%%%%%%%%%%%%%%%%%%%%%%%%%%%%%%%%%%%%%%%%%%%%%
%%%%%%%%%%%%%%%%%%%%%%%%%%%%%%%%%%%%%%%%%%%%%%%%%%%%%%%%%%%%%%%%%%%%%%%%%%

\subsection{Cluster Configuration}

During this phase, the server and clients will be configured to work
together as a cluster.

%%%%%%%%%%%%%%%%%%%%%%%%%%%%%%%%%%%%%%%%%%%%%%%%%%%%%%%%%%%%%%%%%%%%%%%%%%

\subsubsection{Step 5: Complete the cluster configuration}
\label{det:completeinstall}

Press the Step 5 button of the wizard entitled \button{Complete
  Cluster Setup}.  This will do the following:

\begin{enumerate}
\item Queries number of processors from the client nodes.

\item Run the \file{post\_install} script from each OSCAR software
  package.

\item Other final setups and configuration management.
\end{enumerate}

Note that any users created on the server after the OSCAR installation
will not be in the password/group files of the clients until they have
been synced with the server -- you can accomplish this using the C3
\cmd{cpush} tool (see the \cmd{cpush(1)} man page).


%%%%%%%%%%%%%%%%%%%%%%%%%%%%%%%%%%%%%%%%%%%%%%%%%%%%%%%%%%%%%%%%%%%%%%%%%%

\subsubsection{Step 6: Test your cluster using the OSCAR Cluster Test
  software}
\label{det:testcluster}
            
Provided along with OSCAR is a simple test to make sure the key
cluster components (OpenSSH, PBS, MPI, and PVM) are functioning
properly.  Press the Step 6 button of the wizard entitled \button{Test
  Cluster Setup}. This will open a window that will prompt you for the
number clients and number of processors per client. The system's basic
services are checked and then a set of user level tests are run. A
sample dialog is shown in Figure~\ref{fig:detailed-setup-test}. If any
of the test fail, then there is a problem with your installation.

\begin{figure}[htbp]
  \begin{center}
    \includegraphics[scale=\imgscale]{figs/8_test-cluster-complete.\figext}
    \caption{Setup cluster tests}
    \label{fig:detailed-setup-test}
  \end{center}
\end{figure}

%%%%%%%%%%%%%%%%%%%%%%%%%%%%%%%%%%%%%%%%%%%%%%%%%%%%%%%%%%%%%%%%%%%%%%%%%%

\subsubsection{Congratulations!}

Your cluster setup is now complete. Your cluster nodes should
be ready for work.

%%%%%%%%%%%%%%%%%%%%%%%%%%%%%%%%%%%%%%%%%%%%%%%%%%%%%%%%%%%%%%%%%%%%%%%%%%
%%%%%%%%%%%%%%%%%%%%%%%%%%%%%%%%%%%%%%%%%%%%%%%%%%%%%%%%%%%%%%%%%%%%%%%%%%

\subsection{Adding and Deleting client nodes}

This section describes the steps need when it becomes necessary to add
or delete client nodes. If you have already built your cluster
successfully and would like to add or delete a client node, execute
the following from the top-level OSCAR directory:
needs to

\begin{verbatim}
  # ./install_cluster <device>
\end{verbatim}

Like before, you must substitute the device name (e.g., \file{eth1})
for the server node's internal ethernet adapter in the above command.
See Section~\ref{det:installcluster}. Once the OSCAR wizard appears
you are ready to add or delete clients. Note that these steps will
reuse the existing images made with the initial install, however, it
will extend or contract the set of defined clients in the cluster.

%%%%%%%%%%%%%%%%%%%%%%%%%%%%%%%%%%%%%%%%%%%%%%%%%%%%%%%%%%%%%%%%%%%%%%%%%%

\subsubsection{Adding OSCAR clients}

Press the button of the wizard entitled \button{Add OSCAR Clients}. A
sample dialog is shown in Figure~\ref{fig:detailed-add-node}. These
steps should seem familiar -- they are same as steps 3, 4 and 5 of the
initial install. Refer to Sections~\ref{det:defclients},
\ref{det:setupnetwork}, and \ref{det:completeinstall}.

\begin{figure}[htbp]
  \begin{center}
    \includegraphics[scale=\imgscale]{figs/9a_sbs-add-node.\figext}
    \caption{Adding OSCAR clients.}
    \label{fig:detailed-add-node}
  \end{center}
\end{figure}

%%%%%%%%%%%%%%%%%%%%%%%%%%%%%%%%%%%%%%%%%%%%%%%%%%%%%%%%%%%%%%%%%%%%%%%%%%

\subsubsection{Deleting clients}

Press the button of the wizard entitled \button{Delete OSCAR Clients}.
A sample dialog is shown in Figure~\ref{fig:detailed-delete-node}.
Select the node you wish to delete and simply press the button
\button{Delete clients} then press \button{Close}.

\begin{figure}[htbp]
  \begin{center}
    \includegraphics[scale=\imgscale]{figs/10a_sbs-del-node.\figext}
    \caption{Deleting Oscar clients.}
    \label{fig:detailed-delete-node}
  \end{center}
\end{figure}

\endchange
