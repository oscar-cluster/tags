% -*- latex -*-
%
% $Id: sis.tex,v 1.2 2002/07/16 14:39:46 jsquyres Exp $
%
% $COPYRIGHT$
%

\section{Overview of System Installation Suite (SIS)}
\label{sec:sis}

The first question you may have is ``what is SIS?''. The System
Installation Suite (SIS) is a cluster installation tool developed by
the collaboration of the IBM Linux Technology Center and the
SystemImager team.  SIS was chosen to be the installation mechanism
for OSCAR for multiple reasons:

\begin{itemize}
\item SIS is a high-quality, third party, open source product that
  works well in production environments

\item SIS does not require the client nodes to already have Linux
  installed
  
\item SIS maintains a database containing installation and
  configuration information about each node in the cluster

\item SIS uses RPM as a standard for software installation
  
\item SIS supports heterogenous hardware and software installation
  (although this feature is not [yet] used by OSCAR)
\end{itemize}

In order to understand some of the steps in the upcoming install, you
will need knowledge of the main concepts used within SIS. The first
concept is that of an \term{image}. In SIS, an \term{image} is defined
for use by the cluster nodes. This image is a copy of the operating
system files stored on the server. The client nodes install by
replicating this image to their local disk partitions. Another
important concept from SIS it the client definition.  A SIS client is
defined for each of your cluster nodes.  These client definitions keep
track of the pertinent information about each client.  The server node
is responsible for creating the cluster information database and for
servicing client installation requests.  The information that is
stored for each client includes:

\begin{itemize}
\item IP information like hostname, IPaddress, route.
\item Image name.
\end{itemize}

Each of these pieces of information will be discussed further as part
of the detailed install procedure.

For additional information on the concepts in SIS and how to use it,
you should refer to the \file{SIS(1)} man page.  In addition, you can
visit the SIS web site at \url{http://sisuite.org/} for recent
updates.

