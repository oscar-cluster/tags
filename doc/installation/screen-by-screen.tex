% -*- latex -*-
%
% $Id: screen-by-screen.tex,v 1.14 2002/09/02 14:30:04 jsquyres Exp $
%
% $COPYRIGHT$
%

%% Put ourselves on a new page because we go and change the margins for
%% our images
\newpage

\section{Screen-by-Screen Walkthrough}
\label{app:screen-by-screen}

The following is a screen-by-screen walkthrough of a simple installation.
It is intended as supplementary material to aid in providing a better feel
for the general progression of the installation.  For a detailed discussion
of the steps, please refer to the Detailed Cluster Installation Procedure. 

Note the example screen shots were based in a Red Hat 7.3 using a
pre-release version of OSCAR (1.4b7) in the KDE graphical environment.
Since this section is intended as a supplementary source of
information, it is judged to be ``close enough'' to the real 1.4
release.  Just beware that some of the screenshots may not be {\em
  exactly} what you see in the 1.4 release.


%% Adjust the margins so we can better handle our large images

%\setlength{\oddsidemargin}{-0.5in}
%\setlength{\evensidemargin}{-0.5in}
%\setlength{\textwidth}{7.5in}


%------------------------------------------------------------------
% Running install_cluster
%------------------------------------------------------------------

\subsection{Running \cmd{install\_cluster}}

This step comprises of running the \cmd{install\_cluster} script with
the network interface name.  See details in
Section~\ref{det:installcluster}, page~\pageref{det:installcluster}.

\begin{figure}[htbp]
  \begin{center}
    \centerline{\includegraphics[scale=1.0]{figs/0a_sbs-download.\sbsext}}
    \caption{Getting OSCAR.}
    \label{fig:sbs-getting-oscar}
  \end{center}
\end{figure}

% JMS: These images were captured using KDE ``konsole'' tooles, with
% the ``small'' font, using a black-on-white schema, and resized to be
% 80x26.  This allows for 2 screenshots per page.

\begin{figure}[htbp]
  \begin{center}
    \centerline{\includegraphics{figs/0c_sbs-unpack.\sbsext}}
    \caption{Unpacking OSCAR.}
    \label{fig:sbs-unpacking-oscar}
  \end{center}
\end{figure}

\begin{figure}[htbp]
  \begin{center}
    \centerline{\includegraphics{figs/1a_sbs-install-oscar.\sbsext}}
    \caption{Running the \cmd{install\_cluster} script.}
    \label{fig:sbs-install-oscar}
  \end{center}
\end{figure}

\begin{figure}[htbp]
  \begin{center}
    \centerline{\includegraphics{figs/1b_sbs-install-oscar2.\sbsext}}
    \caption{Further progress on running the \cmd{install\_cluster}
      script.}
    \label{fig:sbs-install-oscar2}
  \end{center}
\end{figure}

\begin{figure}[htbp]
  \begin{center}
    \centerline{\includegraphics{figs/2_sbs-oscar-wizard.\sbsext}}
    \caption{The OSCAR Installation Wizard.}
    \label{fig:sbs-install-wizard}
  \end{center}
\end{figure}

\clearpage


%------------------------------------------------------------------
% Step 1: Prepare OSCAR Server For Install
%------------------------------------------------------------------

\subsection{Step 1: Prepare OSCAR Server for Install}

Step 1 is used to setup the server for the OSCAR cluster.  See details
in Section~\ref{det:prepareforinstall},
page~\pageref{det:prepareforinstall}. 

\begin{figure}[htbp]
  \begin{center}
    \centerline{\includegraphics{figs/3a_sbs-wizard-step1.\sbsext}}
    \caption{Beginning step 1 -- Select default MPI implementation.}
    \label{fig:sbs-install-wizard-s1}
  \end{center}
\end{figure}

\begin{figure}[htbp]
  \begin{center}
    \centerline{\includegraphics{figs/3b_sbs-wizard-step1.\sbsext}}
    \caption[Further progress in step 1 -- Prepared server,
    completed.]{Further progress in step 1 -- Prepared server,
      completed (there will also be lots of output in the terminal
      window).}
    \label{fig:sbs-install-wizard-s1b}
  \end{center}
\end{figure}

\clearpage


%------------------------------------------------------------------
% Step 2: Build OSCAR Client Image
%------------------------------------------------------------------

\subsection{Step 2: Build OSCAR Client Image}

Step 2 builds a disk image for the clients to download and install
onto their local disks.  See the details in
Section~\ref{det:buildimage}, page~\pageref{det:buildimage}.

\begin{figure}[htbp]
  \begin{center}
    \centerline{\includegraphics{figs/4a_sbs-build-image1.\sbsext}}
    \caption{Beginning step 2 -- Building the image.}
    \label{fig:sbs-build-image}
  \end{center}
\end{figure}

\begin{figure}[htbp]
  \begin{center}
    \centerline{\includegraphics{figs/4b_sbs-build-image2.\sbsext}}
    \caption[Further progress in step 2 -- Building the
    image.]{Further progress in step 2 -- Building the image; the
      progress bar shows that it is roughly half completed.}
    \label{fig:sbs-build-image-progress}{Further progress in step 2 --
      Building the image.}
  \end{center}
\end{figure}

\begin{figure}[htbp]
  \begin{center}
    \centerline{\includegraphics{figs/4c_sbs-build-image3.\sbsext}}
    \caption{Further progress in step 2 -- Building the image, completed.}
    \label{fig:sbs-build-image2}
  \end{center}
\end{figure}

\clearpage


%------------------------------------------------------------------
% Step 3: Define OSCAR Clients
%------------------------------------------------------------------

\subsection{Step 3: Define OSCAR Clients}

Step 3 is used to specify how many client nodes there will be, and
what their TCP/IP characteristics will be.  See the details in
Section~\ref{det:defclients}, page~\pageref{det:defclients}.
 
\begin{figure}[h]
  \begin{center}
    \centerline{\includegraphics{figs/5a_sbs-define-clients1.\sbsext}}
    \caption{Beginning step 3 -- Defining the clients.}
    \label{fig:sbs-define-clients}
  \end{center}
\end{figure}

\begin{figure}[htbp]
  \begin{center}
    \centerline{\includegraphics{figs/5b_sbs-define-clients2.\sbsext}}
    \caption{Further progress in step 3 -- Create clients for image
      completed.}
    \label{fig:sbs-define-clients2}
  \end{center}
\end{figure}

\clearpage


%------------------------------------------------------------------
% Step 4: Setup Networking
%------------------------------------------------------------------

\subsection{Step 4: Setup Networking}

Step 4 is used to collect the MAC addresses of the client nodes, and
then download the disk images to the client nodes.  See the details in
Section~\ref{det:setupnetwork}, page~\pageref{det:setupnetwork}.

\begin{figure}[h]
  \begin{center}

    \centerline{\includegraphics{figs/6a_sbs-collect-mac1.\sbsext}}
    \caption{Beginning step 4 -- Setting up networking.}
    \label{fig:sbs-setup-network1}
  \end{center}
\end{figure}


%-------
% Show how to create the boot floppy...
%-------

\begin{figure}[htbp]
  \begin{center}
    \centerline{\includegraphics{figs/6ba_sbs-autoinstall-flpy1.\sbsext}}
    \caption{Further progress in step 4 -- Build Autoinstall Floppy
      (part 1).} 
    \label{fig:sbs-autoinstall-flpy1}
  \end{center}
\end{figure}

\begin{figure}[htbp]
  \begin{center}
    \centerline{\includegraphics{figs/6bb_sbs-autoinstall-flpy2.\sbsext}}
    \caption{Further progress in step 4 -- Build Autoinstall Floppy
      (part 2).} 
    \label{fig:sbs-autoinstall-flpy2}
  \end{center}
\end{figure}

\begin{figure}[htbp]
  \begin{center}
    \centerline{\includegraphics{figs/6bc_sbs-autoinstall-flpy3.\sbsext}}
    \caption{Further progress in step 4 -- Build Autoinstall Floppy
      (part 3).} 
    \label{fig:sbs-autoinstall-flpy3}
  \end{center}
\end{figure}

\begin{figure}[htbp]
  \begin{center}
    \centerline{\includegraphics{figs/6c_sbs-client-boot1.\sbsext}}
    \caption{Further progress in step 4 -- Booting the client.}
    \label{fig:sbs-collect-boot1}
  \end{center}
\end{figure}

\begin{figure}[htbp]
  \begin{center}
    \centerline{\includegraphics{figs/6ca_sbs-client-boot1.\sbsext}}
    \caption{Further progress in step 4 -- Scan for booting client (DHCP
      request).}
    \label{fig:sbs-client-boot2}
  \end{center}
\end{figure}

\begin{figure}[htbp]
  \begin{center}
    \centerline{\includegraphics{figs/6d_sbs-broadcast.\sbsext}}
    \caption{Further progress in step 4 -- Client is broadcasting, in
      order to acquire MAC address.}
    \label{fig:sbs-collect-broadcast} 
  \end{center}
\end{figure}

\begin{figure}[htbp]
  \begin{center}
    \centerline{\includegraphics{figs/6e_sbs-found-mac.\sbsext}}
    \caption{Further progress in step 4 -- Scanning network, found
      first MAC address.}
    \label{fig:sbs-setup-network2}
  \end{center}
\end{figure}

\begin{figure}[htbp]
  \begin{center}
    \centerline{\includegraphics{figs/6f_sbs-stop-collect-mac2.\sbsext}}
   \caption[Further progress in step 4 -- Completed setup of
   networking.]{Further progress in step 4 -- Completed setting up of
     networking, got and assigned all MAC address(s).  About to click
     on ``Stop collecting'', and ``Configure DHCP server''.}
    \label{fig:sbs-setup-network3}
  \end{center}
\end{figure}

% Re-adjust the margins so we can better handle our large images
%\setlength{\oddsidemargin}{-0.5in}
%\setlength{\evensidemargin}{-0.5in}
%\setlength{\textwidth}{7.5in}

\begin{figure}[htbp]
  \begin{center}
    \centerline{\includegraphics{figs/6h_sbs-client-boot.\sbsext}}
    \caption{Booting the client a second time to download the image.}
    \label{fig:sbs-install-boot}
  \end{center}
\end{figure}

\begin{figure}[htbp]
  \begin{center}
    \centerline{\includegraphics{figs/6i_sbs-client-diskpar.\sbsext}}
    \caption{Client partitioning disk, setting up the disk tables, and
      starting to download the image.}
    \label{fig:sbs-install-diskpar}
  \end{center}
\end{figure}
  
\begin{figure}[htbp]
  \begin{center}
    \centerline{\includegraphics{figs/6j_sbs-client-rsync.\sbsext}}
    \caption{Client downloading and installing the image.}
    \label{fig:sbs-install-rsync}
  \end{center}
\end{figure}

\begin{figure}[htbp]
  \begin{center}
    \centerline{\includegraphics{figs/6k_sbs-rebootme.\sbsext}}
    \caption{A node has finished the install -- it is asking to be
      rebooted.}
    \label{fig:sbs-install-finish}
  \end{center}
\end{figure}

\clearpage


%------------------------------------------------------------------
% Step 5: Complete Cluster Install
%------------------------------------------------------------------

\subsection{Step 5: Complete Cluster Install} 

Step 5 is used to unify the server and client installations into a
single cluster.  See the details in Section~\ref{det:completeinstall},
page~\pageref{det:completeinstall}.

\begin{figure}[h]
   \begin{center}
     \centerline{\includegraphics{figs/7_sbs-complete-cluster-setup.\sbsext}}
     \caption{Beginning step 5 -- Complete Cluster Setup.}
     \label{fig:sbs-install-wizard-s5}
   \end{center}
 \end{figure}

\clearpage


%------------------------------------------------------------------
% Step 6: Test Cluster Setup
%------------------------------------------------------------------

\subsection{Step 6: Test Cluster Setup}

Step 6 is used to test the cluster setup.  It can either be run from
within the wizard, or, as shown here, from manually launching a shell
script at a \user{root} command prompt.  See the details in
Section~\ref{det:testcluster}, page~\pageref{det:testcluster}.

\begin{figure}[h]
  \begin{center}
    \centerline{\includegraphics{figs/8_sbs-test-cluster-prompt.\sbsext}}
    \caption[Beginning step 6 -- Test cluster setup.]{Beginning step
      6 -- Test cluster setup.  Note that this figure is shown as
      having executed the test script manually, not running from the
      OSCAR wizard GUI.} 
    \label{fig:sbs-install-wizard-s6}
  \end{center}
\end{figure}

\begin{figure}[htbp]
  \begin{center}
    \centerline{\includegraphics{figs/8_sbs-test-cluster-root-tests.\sbsext}}
    \caption{Further progress in step 6 -- Exectuting tests as root.}
    \label{fig:sbs-setup-test}
  \end{center}
\end{figure}

\begin{figure}[htbp]
  \begin{center}
    \centerline{\includegraphics{figs/8_sbs-test-cluster-user-tests.\sbsext}}
    \caption{Further progress in step 6 -- Executing tests as a non-root user.}
    \label{fig:sbs-setup-test1}
  \end{center}
\end{figure}

\begin{figure}[htbp]
  \begin{center}
    \centerline{\includegraphics{figs/8_test-cluster-complete.\sbsext}}
    \caption{Further progress in step 6 -- Cluster tests completed.}
    \label{fig:sbs-setup-test2}
  \end{center}
\end{figure}

\clearpage


%-----------------------------------------------------------------
% Delete Node Button
%-----------------------------------------------------------------

\subsection{Delete Node Button}
\label{app:sbs-delete-node}

The Delete Node button can be used to delete nodes from an OSCAR
cluster.  Note that this button only deletes OSCAR's knowledge of the
nodes -- what physically happens to that node is not OSCAR's concern.
This example shows deleting one of the client nodes setup in the
previous sections -- \hostname{oscarnode2}.  The next section
(Section~\ref{app:sbs-add-node}) will show adding it back.  See the
details on deleting nodes in Section~\ref{det:deleting-clients},
page~\pageref{det:deleting-clients}.

\begin{figure}[h]
  \begin{center}
    \centerline{\includegraphics{figs/10a_sbs-del-node.\sbsext}}
    \caption{Delete OSCAR Clients.}
    \label{fig:sbs-del-node1}
  \end{center}
\end{figure}

\begin{figure}[htbp]
  \begin{center}
    \centerline{\includegraphics{figs/10b_sbs-del-node-partA.\sbsext}}
    \caption{Node selected for deletion.}
    \label{fig:sbs-del-node1-done-partA}
  \end{center}
\end{figure}

\begin{figure}[htbp]
  \begin{center}
    \centerline{\includegraphics{figs/10b_sbs-del-node-partB.\sbsext}}
    \caption{Node deletion completed.}
    \label{fig:sbs-del-node1-done-partB}
  \end{center}
\end{figure}

\clearpage


%------------------------------------------------------------------
% Add Node Button
%------------------------------------------------------------------

\subsection{Add Node Button}
\label{app:sbs-add-node}

The Add Node button will add nodes into an existing OSCAR cluster.  In
this example, we will add back \hostname{oscarnode2} into the cluster.
See the details of adding a node in Section~\ref{det:adding-clients},
page~\pageref{det:adding-clients}.

\begin{figure}[h]
  \begin{center}
    \centerline{\includegraphics{figs/9a_sbs-add-node.\sbsext}}
    \caption{Add OSCAR Clients.}
    \label{fig:sbs-add-node1}
  \end{center}
\end{figure}

\begin{figure}[htbp]
  \begin{center}
    \centerline{\includegraphics{figs/9b_sbs-add-node-mksirange.\sbsext}}
    \caption[Add Node step 1 -- Defining the clients.]{Add Node step 1
      -- Defining the clients.  Note that you only define {\em new}
      clients here (i.e., change the starting number, starting IP,
      etc.).}
    \label{fig:sbs-add-node1-define-clients}
  \end{center}
\end{figure}

\begin{figure}[htbp]
  \begin{center}
    \centerline{\includegraphics{figs/9c_sbs-add-node-success.\sbsext}}
    \caption{Add Node step 1 -- Defining the clients completed.}
    \label{fig:sbs-add-node1-define-clients2}
  \end{center}
\end{figure}

\begin{figure}[htbp]
  \begin{center}
    \centerline{\includegraphics{figs/9d_sbs-add-node-mac1.\sbsext}}
    \caption{Add Node step 2 -- Setting up networking for new node(s).}
    \label{fig:sbs-add-node1-setup-network}
  \end{center}
\end{figure}

\begin{figure}[htbp]
  \begin{center}
    \centerline{\includegraphics{figs/9e_sbs-add-node-mac2.\sbsext}}
    \caption[Add Node step 2 -- Setting up networking, further
    progress]{Add Node step 2 -- Setting up networking, got MAC
      address, stop scanning and configure DHCP server.}
    \label{fig:sbs-add-node1-setup-network2}
  \end{center}
\end{figure}

\begin{figure}[htbp]
  \begin{center}
    \centerline{\includegraphics{figs/9f_sbs-add-node-complete.\sbsext}}
    \caption{Add Node step 3 -- Complete Cluster Setup.}
    \label{fig:sbs-add-node1-cluster-setup}
  \end{center}
\end{figure}

\clearpage

% LocalWords:  tex Exp sbs
