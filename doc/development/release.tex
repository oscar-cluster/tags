% -*- latex -*-
%
% $Id: release.tex,v 1.1 2002/01/23 05:13:41 jsquyres Exp $
%
% $COPYRIGHT$
%

\section{Release Procedures}
\label{sec:release}

OSCAR distribution packages are supervised by a Release Manager.  All
decisions about a release must be approved through the Release
Manager.

\subsection{Approaching a Release}

As a release approaches, the Release Manager will declare the CVS tree
``frozen''.  The following guidelines will be followed after the CVS
tree is frozen until the version is formally released:

\begin{enumerate}
\item Developers will only work on ``show stoppers'' for a release
  that has been frozen.  Show stoppers defined as ``can we ship with
  this bug?''
  
\item Show stoppers will have a priority of 9 for the current release.
  
\item The Release Manager has to approve show stoppers -- i.e., bump
  up to 9 and/or assign someone to work on it.
  
\item Developers must have bugs assigned to you in the tracker before
  working on them.
  
\item Documentation bugs have to be marked as category
  ``documentation'' in the tracker.  Documentation issues and bugs can
  be fixed at any time -- they are not subject to the same ``show
  stopper'' rules as described above.
\end{enumerate}

The Release Manager will determine when a version is ready for
release.

\subsection{Releasing on SourceForge}

Making a release on SourceForge is a little less than obvious.  All of
this is itemized in the main SourceForge documentation; this is an
attempt provide a quick overview that should help release technicians
through the process.

\begin{enumerate}

\item Upload the distribution package file to {\tt
    ftp://upload.sf.net/incoming/}.

\item Login to the SourceForge web page.

\item Go to the ``Admin'' section.

\item Go to the ``Edit/Release Files'' subsection.

{\bf Note:} Stable distribution packages are released under the ``{\tt
  oscar}'' SourceForge package; development/beta distribution packages
  are released under the ``{\tt oscar-devel} SourceForge package.

\item Select ``Add Release'' next to the appropriate SourceForge
  package ({\tt oscar} or {\tt oscar-devel}.

\item Type in the version number in the ``New release name'' text
  box.  It should correspond to the version number of the OSCAR
  distribution package.  For example, ``{\tt 1.1}'', ``{\tt 1.2b7}'',
  and ``{\tt 1.2}''.

\item There are four steps to finish the release:

  \begin{enumerate}

    \item Meta information
      \begin{itemize}
      \item Ensure that the text fields for Release Date and Release
        Name are correct.
        
      \item Set the Status field to ``Active''.
        
      \item Ensure the Of Package field is set to the appropriate
        package.
        
      \item Paste the release text in the ``Paste the Notes In:'' text
        box.
        
      \item Ensure that the ``Preserve my pre-formatted text''
        checkbox is {\em not} checked.

      \item {\bf IMPORTANT:} Click on the ``Submit/Refresh'' button.
        This saves your work so far.
      \end{itemize}

  \item Add the distribution package file to the release.

    \begin{itemize}
    \item In the Step 2 section, there is a list of files.  Check the
      OSCAR distribution package file.
      
    \item {\bf IMPORTANT:} Click on the ``Add Files and/or Refresh
      View'' button.  This saves your work so far.
    \end{itemize}

  \item Edit files in the release

    \begin{itemize}
    \item Once you have completed the previous step, the OSCAR
      distribution file will appear in step 3.

    \item Ensure that all the data in the fields in step 3 is
      correct.  All fields must be filled in, or the file will not
      show up on the release.
      
    \item {\bf IMPORTANT:} Click on the ``Update/Refresh'' button.
      This saves your work so far.
    \end{itemize}

  \item Email release notice

    \begin{itemize}
    \item Check the ``I'm sure'' checkbox

    \item Click on the ``Send Notice'' button.  This sends a mail to
    all SourceForge users who are monitoring the OSCAR package.
    \end{itemize}
    
  \end{enumerate}
\end{enumerate}

Note that the release will show up on the OSCAR download page
immediately, but the distribution package file will not show up for
some time (usually somewhere between 15-30 minutes).

{\large IMPORTANT NOTICE}: {\em Since you cannot delete releases}, do
{\em not} create a new release if you mess one up.  You cannot put a
release in a state that cannot be fixed -- instead of creating a new
one, just go back and fix your previous mistake (you can edit a
previous release .  If necessary, close
your browser and restart the editing process.

\subsection{Removing Old Releases}

SourceForge does not allow deleting of old releases.  All you can do
is ``Hide'' an old release.  That is, go edit the specific release and
set its status to ``Hide''.  This will immediately remove it from the
OSCAR download page.

Note that this will also remove the number of downloads from the total
downloads sum on the downloads page.  Specifically -- the ``total
downloads'' sum shown on the OSCAR download page only reflects the
total number of downloads of all currently active releases.  The total
number of downloads is still maintained on the stats page, regardless
of which releases are hidden and active.

% LocalWords:  Exp
