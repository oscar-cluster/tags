% -*- latex -*-
%
% $Id: release.tex,v 1.4 2002/02/02 04:27:20 jsquyres Exp $
%
% $COPYRIGHT$
%

\section{Release Procedures}
\label{sec:release}

OSCAR distribution packages are supervised by a Release Manager.  All
decisions about a release must be approved through the Release
Manager.

%%%%%%%%%%%%%%%%%%%%%%%%%%%%%%%%%%%%%%%%%%%%%%%%%%%%%%%%%%%%%%%%%%%%%%%%%%%

\subsection{Approaching a Release}

As a release approaches, the Release Manager will declare the CVS tree
``frozen''.  The following guidelines will be followed after the CVS
tree is frozen until the version is formally released:

\begin{enumerate}
\item Developers will only work on ``show stoppers'' for a release
  that has been frozen.  Show stoppers defined as ``can we ship with
  this bug?''
  
\item Show stoppers will have a priority of 9 for the current release.
  
\item The Release Manager has to approve show stoppers -- i.e., bump
  up to 9 and/or assign someone to work on it.
  
\item Developers must have bugs assigned to you in the tracker before
  working on them.
  
\item Documentation bugs have to be marked as category
  ``documentation'' in the tracker.  Documentation issues and bugs can
  be fixed at any time -- they are not subject to the same ``show
  stopper'' rules as described above.
\end{enumerate}

The Release Manager will determine when a version is ready for
release.

%%%%%%%%%%%%%%%%%%%%%%%%%%%%%%%%%%%%%%%%%%%%%%%%%%%%%%%%%%%%%%%%%%%%%%%%%%%

\subsection{Version Numbers}
\label{sec:release-version-numbers}

The OSCAR version number is stored in the file \file{dist/VERSION}.
The rest of this document refers to the file simply as \file{VERSION}
for simplicity.

Working version numbers in CVS should never equal a version that has
already been publicly released.  As such, the version number in CVS
should be changed as soon as a new version is released.

Additionally, the version number in CVS should probably always be a
beta so as to never confuse the CVS version from an official
distribution package version.  Rationale for this is that users will
use whatever version number is available to report bugs.  Consider the
following scenario: OSCAR \version{x.y} is released and the CVS
version number is incremented to \version{x.(y+1)} immediatedly.
Eventually, OSCAR version \version{x.(y+1)} is released.  Now consider
that there are bugs for \version{x.(y+1)} in the bug tracker.  Are
they for the CVS version (and potentially already fixed before
\version{x.(y+1)} was released) or the officially released version?
If the CVS version number always has a ``beta'' nomenclature, such
ambiguties are much easier to resolve.

Hence, part of the release process is to set the version number to be
an appropriate value for the distribution package and then to
increment/change it to be appropriate for continued development in
CVS.
    
For example, if you are releasing OSCAR version 1.2b7, create the
distribution package file that reflects that number and then bump up
\file{VERSION} to \version{1.2b8}.  If you are releasing 1.2, the
\file{VERSION} file will likely still reflect a beta number, so
manually change it to \version{1.2}, create the distribution (and tag
CVS), and then bump up \file{VERSION} (e.g., to \version{1.2.1b1} or
\version{1.3b1})
    
%%%%%%%%%%%%%%%%%%%%%%%%%%%%%%%%%%%%%%%%%%%%%%%%%%%%%%%%%%%%%%%%%%%%%%%%%%%

\subsection{Releasing on SourceForge}

Making a release on SourceForge is a little less than obvious.  All of
this is itemized in the main SourceForge documentation; this is an
attempt provide a quick overview that should help release technicians
through the process.

\begin{enumerate}

\item Before making a distribution package file, ensure that:

  \begin{enumerate}
  \item You have performed a CVS update, and/or otherwise are {\em
      absolutely sure} that you have the latest version of all the
    files.
  \item The top-level \file{README*} files have been updated
  \item The \file{VERSION} file has been updated
  \item You have re-run \file{autogen.sh} and \file{configure} ({\em
      after} editing the \file{VERSION} file) to propgate the new
    version number properly
  \end{enumerate}

\item Make the distribution package file with ``{\tt make dist}''.

\item Tag the CVS source tree with the version number of the tarball
  that you have just created.  

  \begin{enumerate}
  \item Ensure to tag the right version of the \file{VERSION} --
    \file{VERSION} should reflect the version number of the release
    that is currently being created (remember that \file{VERSION}
    needs to be checked in to be tagged).

  \item The nomenclature of the tag names should be
    \file{rel-MAJOR-MINOR[-RELEASE]}.  Note that beta releases are
    consider part of the RELEASE number.  See
    Table~\ref{tab:release-cvs-tags} for some example CVS tag names.

    \begin{table}[htbp]
      \begin{center}
        \begin{tabular}{|l|l|}
          \hline
          \multicolumn{1}{|c|}{Version} &
          \multicolumn{1}{|c|}{CVS tag} \\
          \hline
          \version{1.2b6} & \file{rel-1-2-b6} \\
          \version{1.2.1b3} & \file{rel-1-2-1b3} \\
          \version{1.2} & \file{rel-1-2} \\
          \version{1.3b4} & \file{rel-1-3-b4} \\
          \hline
        \end{tabular}
        \caption{Example CVS tags for corresponding OSCAR version numbers}
        \label{tab:release-cvs-tags}
      \end{center}
    \end{table}
    
  \item Ensure to use the \cmd{-F} option to CVS's \cmd{tag} command
    to force the tag to be placed on the current version.  This is
    necessary if you need to tag the tree twice with the same tag
    (should only happen if you screw up and need to tag the tree a
    second time with the same tag).
  \end{enumerate}

\item Upload the distribution package file to {\tt
    ftp://upload.sf.net/incoming/}.
  
\item Uploading will likely take a few minutes.  While that is
  occuring, edit the \file{VERSION} file and bump up the [beta]
  version number (see Section~\ref{sec:release-version-numbers}).

\item Login to the SourceForge web page.

\item Go to the ``Admin'' section.

\item Go to the ``Edit/Release Files'' subsection.

{\bf Note:} Stable distribution packages are released under the ``{\tt
  oscar}'' SourceForge package; development/beta distribution packages
  are released under the ``{\tt oscar-devel} SourceForge package.

\item Select ``Add Release'' next to the appropriate SourceForge
  package ({\tt oscar} or {\tt oscar-devel}.

\item Type in the version number in the ``New release name'' text
  box.  It should correspond to the version number of the OSCAR
  distribution package.  For example, ``{\tt 1.1}'', ``{\tt 1.2b7}'',
  and ``{\tt 1.2}''.

\item There are four steps to finish the release:

  \begin{enumerate}

    \item Meta information
      \begin{itemize}
      \item Ensure that the text fields for Release Date and Release
        Name are correct.
        
      \item Set the Status field to ``Active''.
        
      \item Ensure the Of Package field is set to the appropriate
        package.
        
      \item Paste the release text in the ``Paste the Notes In:'' text
        box.  It is preferable to use HTML in the release text to
        ensure proper formatting of the notice when it is displayed on
        a user's browser.
        
      \item Ensure that the ``Preserve my pre-formatted text''
        checkbox is {\em not} checked.  {\bf Note:} If you did not use
        HTML in the release notes, you may wish to check this option.
        If you do, ensure that paragraphs do not contain any line
        breaks so that each paragraph is one long line (to ensure
        proper formatting of the notice when it is displayed on a
        user's browser -- particularly since you just pasted the text
        into the text box.

      \item {\bf IMPORTANT:} Click on the ``Submit/Refresh'' button.
        This saves your work so far.
      \end{itemize}

  \item Add the distribution package file to the release.

    \begin{itemize}
    \item In the Step 2 section, there is a list of files.  Check the
      OSCAR distribution package file.
      
    \item {\bf IMPORTANT:} Click on the ``Add Files and/or Refresh
      View'' button.  This saves your work so far.
    \end{itemize}

  \item Edit files in the release

    \begin{itemize}
    \item Once you have completed the previous step, the OSCAR
      distribution file will appear in step 3.

    \item Ensure that all the data in the fields in step 3 is
      correct.  All fields must be filled in, or the file will not
      show up on the release.
      
    \item {\bf IMPORTANT:} Click on the ``Update/Refresh'' button.
      This saves your work so far.
    \end{itemize}

  \item Email release notice

    \begin{itemize}
    \item Check the ``I'm sure'' checkbox

    \item Click on the ``Send Notice'' button.  This sends a mail to
    all SourceForge users who are monitoring the OSCAR package.
    \end{itemize}
    
  \end{enumerate}
  
\item Update the OSCAR web pages to contain the version number and
  link to download the latest version of the OSCAR distribution
  package.  You may need to wait until the distribution package file
  appears on the SourceForge files page.
  
\end{enumerate}

Note that the release will show up on the OSCAR download page
immediately, but the distribution package file will not show up for
some time (usually somewhere between 15-30 minutes).

{\large IMPORTANT NOTICE}: {\em Since you cannot delete releases}, do
{\em not} create a new release if you mess one up.  You cannot put a
release in a state that cannot be fixed -- instead of creating a new
one, just go back and fix your previous mistake (you can edit a
previous release .  If necessary, close
your browser and restart the editing process.

%%%%%%%%%%%%%%%%%%%%%%%%%%%%%%%%%%%%%%%%%%%%%%%%%%%%%%%%%%%%%%%%%%%%%%%%%%%

\subsection{Removing Old Releases}

SourceForge does not allow deleting of old releases.  All you can do
is ``Hide'' an old release.  That is, go edit the specific release and
set its status to ``Hide''.  This will immediately remove it from the
OSCAR download page.

Note that this will also remove the number of downloads from the total
downloads sum on the downloads page.  Specifically -- the ``total
downloads'' sum shown on the OSCAR download page only reflects the
total number of downloads of all currently active releases.  The total
number of downloads is still maintained on the stats page, regardless
of which releases are hidden and active.

% LocalWords:  Exp
