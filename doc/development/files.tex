% -*- latex -*-
%
% $Id: files.tex,v 1.3 2002/01/23 05:13:41 jsquyres Exp $
%
% $COPYRIGHT$
%

\section{File Conventions}
\label{sec:file-conventions}

The following conventions will be used for all files (including
directories) within the OSCAR development tree and distribution
package:

\begin{enumerate}
  
\item All files (excluding directories) should have an extension.
  Table~\ref{tbl:files-exts} lists file types and their extensions.
  Some files may be exempt from the rule, and will be dealt with on a
  case-by-case basis.  Some obvious exceptions include the {\tt
    README}, {\tt INSTALL}, and {\tt COPYING} files.

  \begin{table}[tb]
    \begin{center}
      \begin{tabular}{|l|l|}
        \hline
        \multicolumn{1}{|c|}{{\bf File type}} & 
        \multicolumn{1}{c|}{{\bf Extension}} \\
        \hline
        Bourne / Bash shell script & {\tt .sh} \\
        Perl script & {\tt .pl} \\
        Text file & {\tt .txt} \\
        \hline
        Compressed tarball & {\tt .tar.Z} \\
        Gzipped tarball & {\tt .tar.gz} \\
        Bzipped tarball & {\tt .tar.bz2} \\
        \hline
        \LaTeX\ source code & {\tt .tex} \\
        {\tt xfig} figures & {\tt .fig} \\
        Bibtex source code & {\tt .bib} \\
        Postscript files & {\tt .ps} \\
        Encapsulated postscript files & {\tt .eps} \\
        PDF files & {\tt .pdf} \\
        \hline
      \end{tabular}
      \caption[File types and their associated filename 
        extensions.]{File types and their associated filename
        extensions.  Note that ``{\tt .tgz}'' is not used.}
      \label{tbl:files-exts}
    \end{center}
  \end{table}
  
\item The name ``oscar'' will not appear in directory names.  This is
  redundant when everything is already under a top-level ``oscar''
  directory.
  
\item Generally, ``oscar'' should not need to appear in filenames,
  either.  But there are some obvious exceptions, such as ``{\tt
    oscar-wizard}'', part of a third party's package filename, etc.
  
\item Dashes should be used to separate multiple words in a directory
  name.  Multi-word directory names are to be of the form:
  first-second-third.  All letters in a directory name should be lower
  case (e.g., ``{\tt packages/pbs}'').

\item Likewise, dashes should be used to separate mulitple words in a
  filename.  For example, {\tt third-party.tex}.
  
\item Package files that were created specifically for OSCAR should
  indicate that fact in their filenames.  A Mandrake-like convention
  seems good for RPMs; tarballs can include the string ``{\tt oscar}''
  right before ``{\tt .tar.gz}''.  For example:

  \begin{itemize}
  \item {\tt lam-{\rm X}.{\rm Y}.{\rm Z}-1oscar.i686.rpm}
  \item {\tt lui-{\rm A}.{\rm B}.{\rm C}oscar.tar.gz}
  \end{itemize}

\item All executable scripts should have their execute bit set.
  
\item All text files (scripts included) should include the CVS token
  ``{\tt \$Id\$}'' within the first few lines of the file so that the
  last editing user / timestamp is automatically maintained.
  
\item All text files (scripts included) should include the ``{\tt
    \$COPYRIGHT\$}'' token where the copyright/license notice can be
  inserted during ``{\tt make dist}'' time.

\end{enumerate}

% LocalWords:  Exp sh pl txt gz gzipped tarballs tgz bz oscar izard RPMs dist
% LocalWords:  OPYRIGHT
