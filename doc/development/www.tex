% -*- latex -*-
%
% $Id: www.tex,v 1.2 2003/06/18 00:17:59 naughtont Exp $
%
% $COPYRIGHT$
%

\section{Web Maintenance}
\label{sec:www}

The OSCAR web pages are maintained in CVS.  The following section will
outline the steps involved in management of the web material.  This will
include the necessary steps to checkout the ``{\tt www}'' module from the
OSCAR CVS repository and later make the changes ``live'' on 
\url{http://oscar.sourceforge.net/}.   Note, the abbreviated form of
\url{sourceforge.net} is often used in the instructions, i.e., \url{sf.net}.

%%%%%%%%%%%%%%%%%%%%%%%%%%%%%%%%%%%%%%%%%%%%%%%%%%%%%%%%%%%%%%%%%%%%%%%%%%

\subsection{Pre-requirements}

As mentioned above the OSCAR projects web pages are located in the CVS
module ``{\tt www}''.  In order to perform maintenance on the documents one
must be able to connect to \url{cvs.oscar.sourceforge.net} and (aka
\url{cvs.oscar.sf.net}) via \cmd{ssh}.  This may be an issue for developers
behind firewalls. 

Also, you must have an OSCAR developer account in order to commit changes
to the repository.  You can send mail to the
\listname{oscar-devel@lists.sourceforge.net} to request an account if
necessary.



%%%%%%%%%%%%%%%%%%%%%%%%%%%%%%%%%%%%%%%%%%%%%%%%%%%%%%%%%%%%%%%%%%%%%%%%%%

\subsection{Overview of Steps}

The steps involved in making changes to the OSCAR web pages involves
two basic parts: (1) obtaining the material from CVS to the local
``editing'' machine, (2) updating the ``online'' files at
\url{oscar.sf.net}.  The set of steps involved include:

\begin{enumerate}
	\item Obtain (checkout) the material from CVS to the ``editing'' machine
	\item Make changes and update CVS (update, commit)
	\item Connect to web server \url{oscar.sf.net}
	\item Update (update) the ``online'' copy for all changed files
	\item Change permissions of new (update'd) files to be group writable
	(\cmd{chmod g+w})
\end{enumerate}



%%%%%%%%%%%%%%%%%%%%%%%%%%%%%%%%%%%%%%%%%%%%%%%%%%%%%%%%%%%%%%%%%%%%%%%%%%

\subsection{Detail of Steps}

The process of checking in/out from CVS is probably redundant for many but
the following details an example usage for those less familiar.  Note, there
are many useful resources on the web as well as off the Source Forge
project page for using CVS.  You are encouraged to read over the CVS(1)
manual page.   In the following examples, the abbreviated form of the
\cmd{cvs} commands are also provided in parentheses, e.g., \cmd{checkout}
(co).

\begin{enumerate}
	\item In the following examples the developer's username is
	\emph{sgrundy}~\footnote{Solomon Grundy want OSCAR too!}.

	\item Set a few necessary environment variables.
	\begin{verbatim}
		shell: $  export CVS_RSH=ssh
		shell: $  export CVSROOT=sgrundy@cvs.oscar.sf.net:/cvsroot/oscar
	\end{verbatim}

	\item Perform a CVS checkout (co) of the {\tt www} module,
	\begin{verbatim}
		shell: $  cvs co -d oscar-www www
		sgrundy@cvs.oscar.sf.net's password: 
		U oscar-www/.cvsignore
		U oscar-www/docs.php
		U oscar-www/favicon.ico
		U oscar-www/index.php
		...
	\end{verbatim}

	\item Edit the exiting files or add any new
	files/directories~\footnote{To add a new file/dir to CVS use: \cmd{cvs
	add \emph{FILE}}.}

	\item After making the changes update your checked out copy to obtain
	any changes others might have made.  The~{\tt -dP}~options are helpful
	to explicitly state a directory (-d  .) and to prune directories or
	files that may be empty or stale (-P).
	\begin{verbatim}
		shell: $  cvs update -dP .
		M index.php
		cvs server: Updating docs
		cvs server: Updating errata
		...
	\end{verbatim}
	Note, CVS checks for changes recursively so it may be better to list
	specific files or a directory to limit the scope of things checked.  
	\begin{verbatim}
		shell: $  cvs update  index.php
		M index.php
	\end{verbatim}

	\item At this point only changes to the local checked out copy of the
	module have been made through the \cmd{cvs update}.  The following
	command will write (commit) the changes to the actual CVS repository.
	Remember this is recursive so you may want to list explicit files/dirs
	or change to the top of the relevant sub-directory containing changes.

	Provide a descriptive log message for the changes being committed!  And
	type \cmd{:wq} to save and quit changes from 'vi' (default editor).
	\begin{verbatim}
		shell: $  cvs commit index.php
		* Added the new introduction text and removed the old OSCAR
		  Symposium text.
		CVS: ------------------------------------------------------------------
		CVS: Enter Log.  Lines beginning with `CVS:' are removed automatically
		CVS:
		CVS: Committing in .
		CVS:
		CVS: Modified Files:
		CVS:    index.php
		CVS: ------------------------------------------------------------------

		Checking in index.php;
		/cvsroot/oscar/www/index.php,v  <--  index.php
		new revision: 1.12; previous revision: 1.11
		done
		Mailing oscar-checkins@lists.sourceforge.net...
		Generating notification message...
		Generating notification message... done.
	\end{verbatim}

	\item The changes have been added to CVS and now need to be updated in
	the ``live'' online copy.  This requires you to log into the web server
	using \cmd{ssh}.  Then you will change to the appropriate directory and
	update the checkout on the web server, e.g., update the \file{index.php}.
	\begin{verbatim}
		shell: $  ssh oscar.sf.net
		sgrundy@oscar.sf.net's password:

		sgrundy@sc8-pr-shell1:~$  cd oscar/htdocs/
		sgrundy@sc8-pr-shell1:~/oscar/htdocs$ cvs update index.php
		sgrundy@cvs1's password: 
		P index.php
	\end{verbatim}

	\item {\bf *IMPORTANT*} You must now set any updated/new files to have
	group writable permissions, e.g., \file{index.php}~\footnote{This is a
	problem related to the CVS permissions not being honored, if you forget
	this step no one else will be able to change this file.  It shouldn't
	hurt to \cmd{chmod g+w *} since all files should have this bit set.
	Regardless, it is likely that others will not be happy with you if you
	forget this step. :-)}.
	\begin{verbatim}
		sgrundy@sc8-pr-shell1:~/oscar/htdocs$ chmod g+w index.php
	\end{verbatim}

\end{enumerate}



