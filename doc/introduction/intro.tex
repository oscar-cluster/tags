% -*- latex -*-
%
% $Id: intro.tex,v 1.3 2001/12/21 08:22:50 ngorsuch Exp $
%
% $COPYRIGHT$
%

\documentclass[letterpaper,11pt]{article}

% -*- latex -*-
%
% $Id: defs.tex,v 1.16 2003/02/24 23:30:37 naughtont Exp $
%
% $COPYRIGHT$
%

\usepackage{times}
\usepackage{graphicx}
\usepackage{fullpage}
\usepackage{verbatim}
\usepackage{url}

% Versions

\include{version}

% Image Scaling

\def\imgvskip{5pt}
\def\imghskip{0.125in}


% PDF stuff
 
\newif\ifpdf
\ifx\pdfoutput\undefined
   \pdffalse
\else
   \pdfoutput=1
   \pdftrue
\fi
 
\setcounter{tocdepth}{3}
 
\ifpdf
  \usepackage[pdftex,
              colorlinks=true,
              linkcolor=blue,filecolor=blue,pagecolor=blue,urlcolor=blue
              ]{hyperref}
  \def\imgscale{0.5}
\else
  \usepackage[colorlinks=true,
              linkcolor=blue,filecolor=blue,pagecolor=blue,urlcolor=blue
              ]{hyperref}
  \def\imgscale{1}
\fi

% Stoopid European-based Linux distributions...
% Gotta force letter size paper, not A4

\ExecuteOptions{letterpaper}

% Discussion item; stolen from the MPI standard

\newenvironment{discuss}{\begin{list}{}{}\item[]{\it Discussion item:}
\addcontentsline{toc}{subsection}{Discussion Item}
}{{\rm ({\it End of discussion item.})} \end{list}}

% Rationale item; stolen from the MPI standard

\newenvironment{rationale}{\begin{list}{}{}\item[]{\it Rationale.}
}{{\rm ({\it End of rationale.})} \end{list}}

% Stolen from the MPI standard

\newcommand{\cmd}[1]{\texttt{#1}}
\newcommand{\env}[1]{\texttt{#1}}
\newcommand{\odrcat}[1]{\textsf{#1}}
\newcommand{\odrkey}[1]{\texttt{#1}}
\newcommand{\file}[1]{\texttt{#1}}
\newcommand{\rpmname}[1]{\texttt{#1}}
\newcommand{\user}[1]{\texttt{#1}}
\newcommand{\hostname}[1]{\texttt{#1}}

\newcommand{\term}[1]{\textit{#1}}

\newcommand{\msg}[1]{\textbf{#1}}
\newcommand{\msgout}[1]{\textbf{\texttt{#1}}}

\newcommand{\panel}[1]{\textbf{\texttt{#1}}}
\newcommand{\button}[1]{\textbf{\texttt{$<$#1$>$}}}
\newcommand{\field}[1]{\textit{\texttt{#1}}}

\def\optional{\noindent {\em Note: This step is optional.}\vspace{11pt}}

% Also stolen from the MPI standard
% intended for general change marks not associated with a certain version
\def\begchange{\marginpar[\hspace*{-60pt}\mbox{\hspace*{10pt}
$\top$ \tiny (General)}]{\mbox{$\top$ \tiny (General)}}}
\def\endchange{\marginpar[\hspace*{-60pt}\mbox{\hspace*{10pt}
$\bot$ \tiny (General)}]{\mbox{$\bot$ \tiny (General)}}}
%get rid of these change marks
%\def\begchange{}
%\def\endchange{}


\begin{document}

\title{OSCAR: A packaged Cluster software stack for \\
       High Performance Computing} 
\author{The Open Cluster Group \\
\url{http://www.openclutergroup.org/}}

\date{December 13, 2001}

\maketitle

\abstract{OSCAR is a fully integrated easy to install bundle of
software designed to make it easy to build and use a cluster for high
performance computing.  Everything you need to build, maintain, and
use a modest sized Linux cluster is included in OSCAR.  In this note,
we introduce OSCAR and provide the background information you need to
use it.}


\section{Introduction}
OSCAR is a package of RPM's, perl-scripts, libraries, tools, and
whatever else is needed to build and use a modest-sized Linux cluster.
With OSCAR, you don't need to roam the web to find what you need.
Just download a single package, install it, and you're ready to start
using your cluster.

The acronym, OSCAR, is a bit contrived, but it goes a long way towards
explaining what the goal of the OSCAR project is: 
\begin{center}
 \underline{O}pen \underline{S}ource \underline{C}luster 
 \underline{A}pplication \underline{R}esources
\end{center}

First and foremost, OSCAR is an Open Source project.  Every component
within OSCAR is available under one of the well known Open Source
licenses (e.g. GPL).  The goal of OSCAR is making a cluster easy to
build, easy to maintain and easy to use.  In other words, OSCAR
contains the resources you need to apply cluster computing to your
High Performance Computing problems.

Now that you know what OSCAR is, lets be absolutely clear about what
OSCAR is not.  OSCAR is not a new standard.  It is not an attempt to
cram a particular approach to clustering down the collective throats
of the high performance computing community.

Rather, OSCAR is a snapshot of current, \emph{best-known-practices} in
cluster computing.  The creators of OSCAR -- the Open Cluster Working
Group -- have studied what works at cluster-computing sites around the
world, and gathered it into a single integrated package.  our hope is
that with OSCAR it will be easy for people to replicate successful
clustering techniques at their own sites.

OSCAR is not unique.  The Beowulf and extreme-Linux projects have
essentially implemented the same idea.  What distinguishes OSCAR is
its continuity.  The Open Cluster Group will continue to update the
package and make sure it represents a current snapshot of
best-known-practices for building and using clusters.

Another unique feature of OSCAR is its origins from a collaboration of
hardware vendors, software vendors, and national-labs.  The national
labs are a great source of experience and open source technology.
Software vendors bring software expertise, but also channels for fully
supported instantiations of OSCAR.  Finally, the hardware vendors
bring exposure to OSCAR and will be able to propagate it widely.

In these notes, we will describe what a typical OSCAR cluster looks
like.  We will then describe at a very high level each of the
components in the current release of OSCAR.

\section{OSCAR Cluster: hardware considerations}
What does an OSCAR cluster look like?  We expect clusters built with
OSCAR to vary considerably from one site to another on local policies
or the needs of the cluster's users.

To help clarify the discussion that follows, we need to define a
canonical OSCAR cluster.  An OSCAR cluster consists of Servers, a
gateway, nodes, and a network.  Lets look at each one of these.

\begin{description}
        \item[Servers:] As the name implies, \emph{servers} are computers
        that provide services to the cluster.  This includes the NFS file
        server, the PBS server and any other server-functions required by
        OSCAR.

        \item[Gateway:] The gateway has at least two network connections.
        One goes to the cluster's internal network.  The second connection
        goes to an external local area network.  By using a gateway, a
        cluster administrator can choose to relax security constraints
        inside the cluster.  Since nodes can only leave the cluster
        through the gateway, you can put full security measures on the
        gateway's external LAN connection and protect the cluster behind
        it.  This is an optional component of an OSCAR cluster, though we
        find it hard to believe a production cluster could meet the needs
        of an organization without a gateway.

        \item[Nodes:] The nodes are the heart of a working cluster -- the
        computers that do the actual computing.  There must be at least
        one node, though more typically there will be somewhere between 4
        and 100 nodes.  The nodes must have local disk storage.  Note that
        nothing in OSCAR precludes its use for huge clusters with many
        hundreds of nodes, but the package has not been tuned with
        arbitrarily large numbers of nodes in mind.  The nodes can be
        different from each other (resulting in a heterogeneous cluster),
        but in most cases, we expect the nodes will be similar (i.e. the
        same CPU architecture and comparable memory sizes and speeds).

        \item[Network:] Every cluster must have at least one network: i.e.
        some way of connecting its computers together.  OSCAR currently
        requires that an Ethernet network connects the computers
        comprising the cluster.  There may be an additional network to
        provide high performance communication, but this network will not
        be used to install cluster software.

\end{description}

While it isn't required in principle, we combine all servers within
OSCAR onto a single node.  It could be any node, but to keep things as
simple as possible, we use one computer to be both our gateway and our
server node.

\section{OSCAR Software Components}
OSCAR is a collection of integrated software components that are used
to build, maintain and use a cluster.  This big job can be broken down
into a number of core functionalities:

\begin{itemize}
        \item Installing Linux on each node

        \item Building a database of the cluster and then
        semi-automatically installing OSCAR.

        \item Security

        \item Cluster management

        \item Setting up the libraries and tools needed to build programs
        to run on the cluster

        \item Workload management tools for multi-user clusters -- batch
        queues, scheduling, and job monitoring.

        \item Packaging and documentation
\end{itemize}

In the following sub-sections, we describe the OSCAR components that
implement each of these functionalities.  We restrict ourselves to a
high level overview of each package since more detailed documentation
is included in the OSCAR distribution.


\subsection{Linux installation: SIS}
SIS \cite{SIS} the \emph{System Installation Suite}, is a
Linux installation tool co-developed by the LUI team from IBM and 
the SystemImager\cite{SI}  team. It is released under the GPL.
It is used to install heterogeneous Linux clusters over a network.

SIS uses an image based model. That is, a copy of all the files that
are installed on the nodes resides on the server. When the nodes install
they use rsync to pull these files over to their local disks. This provides
an option for maintenance as well since you can update the image on the server
and have all the nodes update to match that image.

We also use SIS to build a database describing the cluster itself.
This database includes the node names, network information, cluster
configuration data, and anything else required to install the other
components of OSCAR.


\subsection{Security: OpenSSH}
A cluster usually sits in a ``back-room'' with users coming in over a
LAN connection to the Gateway node.  When everything is sitting behind
a firewall and the level of trust between users is high, additional
security measures may not be required.  As soon as the cluster is
connected to larger networks, however, all traffic to and from the
cluster should be encrypted.  If users need security from other users,
then ssh can be used for inter-cluster instructions as well.

The most common way to allow secure connections in a Linux environment
is with OpenSSH \cite{OpenSSH}.  OpenSSH is a collection of packages
that handle secure connections, server-side SSH services, secure-key
generation and any other functions required to support secure
connections between computers.  We provide and configure OpenSSH 
within OSCAR. 

\subsection{Cluster management: C3}
Each computer in a cluster runs its own copy of the operating system.
In many ways, this is a ``good thing'': since it allows you to build a
high performance supercomputer using ``off the shelf'' operating
systems designed for the mass-market.  The problem is that for certain
operations, you want to view the cluster as a single computer.  Files
need to be moved around the computer, processes started on groups of
nodes, and other system-wide  operations we take for granted on a
single computer.

We call these operations ``cluster management''.  The cluster
management package used in OSCAR is C3 \cite{C3} from Oak Ridge
National Laboratory.


\subsection{Programming environments: MPI and PVM}
Most users of cluster write the software that runs on the cluster.
There are many different ways to write software for clusters.  The
most common approach is to use a message passing library.  We have
included MPI (Message Passing Interface, both the LAM/MPI~\cite{lam}
and MPICH~\cite{mpich} implementations) and PVM~\cite{PVM} (Parallel
Virtual Machine).  At this time, compilers or math libraries installed
by OSCAR come from the Linux distribution.


\subsection{Workload Management: PBS}
When multiple people share a cluster, some type of workload management
is needed.  Actually, even one person running a large mixture of jobs
may need help managing the work.  A workload management system ensures
that every person (or job) gets their fair share of the cluster, and
all resources get used efficiently.

There are three distinct components to workload management: resource
management, job management, and job scheduling.  For our workload
management software, we use PBS \cite{OpenPBS}, the Portable Batch System
from Veridian.  Once you queue jobs to PBS, PBS monitors the state of
the cluster, and handles starting (and stopping) jobs, and delivery of
output.  Effectively scheduling jobs on the cluster requires a job
scheduler.

The Maui scheduler~\cite{MAUI} is used for scheduling jobs in PBS.

\section{OSCAR \oscarversion: Release Notes}
In release \oscarversion\ of OSCAR, we include the following specific
packages:
\begin{itemize}
        \item C3 - 2.7
        \item LAM/MPI - 6.5.5
        \item SIS - 1.0
        \item Maui Scheduler - 3.0.6p4
        \item MPICH - 1.2.1 (from MCS Software based upon MPICH 1.2.1)
        \item OpenSSH - 2.5.2p2
        \item OpenSSL - 0.9.6
        \item PBS - 2.2p11
        \item PVM - 3.4.3
\end{itemize}

In each case, we include the full package including source code and
associated documentation.  Information about how to use each of the
above packages can be found in the distribution directory for the
package in question.

In addition to the above packages, OSCAR \oscarversion\ includes an
installation wizard based closely on the work by the SIS team on a GUI
for SIS.  After loading the OSCAR distribution onto the server node, a
shell script called \cmd{install\_cluster} is run.  This will bring up
the installation wizard, which will guide you through each step of
building an OSCAR cluster.

OSCAR has been validated for RedHat 7.1.  It should be straightforward
to use OSCAR with other Linux distributions, but we have not tested
this yet.

\section{Conclusion and Future Plans}
We have big plans for OSCAR.  There are other components we are
looking into adding to OSCAR:
\begin{itemize}
        \item We are exploring various GUIs to provide an easy to use
        single point configuration.

        \item There are still too many steps in building an OSCAR cluster
        that are manual.  We are going to continuously strive to automate
        OSCAR.

        \item We need to expand the OSCAR procedures to make it easier to
        deviate from the canonical cluster.  For example, a user should be
        able to easily omit installation of packages they don't plan to
        use (for example, a single-user cluster may not need PBS).
\end{itemize}

In addition to the above specific plans, we are going to explore ways
to extend OSCAR to a broader range of clusters.  For example, it might
be nice to use OSCAR for building high availability clusters.  Another
possibility if to extend OSCAR to handle diskless nodes.

In addition to changes in OSCAR itself, we have big plans for the
project overall.  We want to see OSCAR and the Open Cluster Group grow
into a major force in cluster computing.  We want to see OSCAR become
a starting point that companies will use to build supported cluster
software stacks.  We want academics interested in tools-research to
use OSCAR as the core software type build their tools on top of.

Computer scientists spend way too much time ``re-inventing the
wheel''.  We hope that OSCAR can help put an end to this re-inventing
-- at least in terms of the basic components of an open source
cluster.

\section{Acknowledgments}
The Open Cluster Group includes representatives from Dell Computers,
IBM, Intel Corporation, the National Center for Supercomputing
Applications, Oak Ridge National Laboratory, MSC-software, SGI, and
Veridian Corporation.  Key contributors to the project are:
\begin{itemize}
        \item Gabriel Broner and John Hesterberg of SGI

        \item Rich Ferri, Michael Chase-Salerno and Sean Dague of IBM

        \item David Lombard of MSC-software

        \item Tim Mattson of Intel

        \item Jenwei Hsieh of Dell

        \item Bill Nitzberg and Bhroam Mann of Veridian

        \item Rob Pennington, Jeremy Enos, Neil Gorsuch of NCSA
        
        \item Stephen Scott, Brian Luethke, and Mike Brim of ORNL

        \item Jeff Squyres from Indiana University

        \item Brian Finley from Bald Guy Software
\end{itemize}
\noindent Of these people Rich Ferri, Michael Chase-Salerno, Sean Dague,
David Lombard, Tim Mattson, Bhroam Mann, Jeremy Enos, Neil Gorsuch, Jeff
Squyres, Stephen Scott and Mike Brim did the actual hard work of
finding and packaging the software inside OSCAR.

PBS includes software developed by NASA Ames Research Center, Lawrence
Livermore National Laboratory, and Veridian.

\bibliographystyle{plain}
\bibliography{intro-refs}

\appendix

\ \\

\section*{Glossary}
The definitions in this glossary are specialized to their use in
OSCAR.
\begin{description}
        \item[Boot kernel:] The kernel that is sent over the network to
        initially boot the node, mount the remote root file system, and
        prepare the hard-drive for installation.

        \item[Linux Kernel:] The kernel that is permanently installed on
        the hard-drive of the node, and is used to boot the node after
        installation.

        \item[MAC Address:] A 6 byte hex address that is uniquely assigned
        to each ethernet adapter.  The MAC address is used to identify
        individual nodes during the boot process.

        \item[RAM disk:] The RAM disk is used to configure adapters that
        are not directly supported in the Linux kernel.  Refer to the
        Linux \cmd{mkinitrd} command for further details.

        \item[RPM:] RedHat Package Manager.  RPM is the most widely used
        Linux method of packaging software so that it checks for pre-reqs
        and co-reqs during installation of the package.

        \item[Image:] The copy of the operating system files for the nodes
        that is stored on the server. This is propogated to the nodes using
        SIS.
\end{description}


\end{document}

