% $Id: pkg-layout.tex,v 1.15 2003/10/06 22:26:19 naughtont Exp $

% TJN: Mention the XML tools that are helpful, e.g., 'xmllint'
%      which >= 2.5.8 (i believe) has --schema option too! :)
%      Also, probably want to include the DTD (& Schema) for
%      general perusal/clarity.

\section{Package Layout}
\label{sect:pkg-layout}

As OSCAR evolved it became obvious that the mechanism to configure and
install a cluster needed to be cleanly separated from the software that was
to be installed.  The approach taken was to create \emph{OSCAR Packages}.
The OSCAR Package layout is geared toward making things as simple as
possible for package authors.  So, in its simplest form an OSCAR Package is
an RPM\footnote{A binary RPM compiled for an OSCAR supported Linux
distribution.}.  However, most software requires further configuration for
a cluster environment so additional scripts, documentation, etc. may be
added.  The basic directory structure for an OSCAR Package is as follows.

\begin{quote}
\begin{description}
  \item[\file{config.xml}] -- meta file with description, version, etc.
  \item[\directory{RPMS/}] -- directory containing binary RPM(s) for the package
  \item[\directory{SRPMS/}] -- directory containing source RPM(s) used to build
                            the package
  \item[\directory{scripts/}] --  set of scripts that run at particular times
                     during the installation/configuration of the cluster
  \item[\directory{testing/}] -- unit test scripts for the package
  \item[\directory{doc/}] -- documentation and/or license information
\end{description}
\end{quote}

\noindent For reference purposes, Table~\ref{tab:oscar-envvars}
contains a list of the currently recognized environment variables used by
the OSCAR framework.

% Table with all currently supported/recognized Environment Variables
%  $Id: oscar-envvars-table.tex,v 1.4 2003/09/26 18:58:08 naughtont Exp $


\begin{table}[htbp]
  \begin{center}
  \begin{tabular}{|l|l|} \hline
  {\bfseries Environment Variable} & {\bfseries Description} \\\hline
  \hline
  \envvar{OSCAR\_HOME} & 
       Defines top-level directory for OSCAR installation 
	   \\\hline
%
  \envvar{OSCAR\_PACKAGE\_HOME} & 
       Points to a packages directory (used with OPD) 
	   \\\hline
%
  \envvar{OSCAR\_HEAD\_INTERNAL\_INTERFACE} & 
       Contains the value provided to 'install\_cluster \emph{ethX}' 
	   \\\hline
%%
%TJN: I think this should be added, not currently a valid EnVar
%  \envvar{OSCAR\_RPMPOOL} & 
%       Defines an alt. location for RPM (default = \directory{/tftpboot/rpm})  
%	   \\\hline
%
  \end{tabular}
  \caption{Environment variables currently recognized/used by OSCAR.} 
  \label{tab:oscar-envvars}
  \end{center}  
\end{table}


The packaging API provides authors the ability to make use of configuration
scripts to setup cluster software outside of the RPM itself.  The scripts
fire at different stages of the installation process as detailed in
Section~\ref{sect:pkg-scripts}.  The packages may also include simple test
scripts in the \directory{testing/} directory, which are used to verify the
software was properly installed (see Section~\ref{sect:pkg-testing}).
Lastly, an OSCAR Package Downloader (OPD) tool is provided to simplify
acquisition of new packages (see Section~\ref{sect:opd}).




\subsection{\file{config.xml}}
\label{sect:pkg-config-xml}

This XML file provides the package name (\xmltag{name}), version number
(\xmltag{version}) and description information (\xmltag{description}) as
well as the list of RPMS (\xmltag{rpmlist}) and their installation location
(e.g., ``oscar\_server'', ``oscar\_clients'').   The file enables an author
to convey constraints such as supported distribution (\xmltag{filter}) or
simple dependencies (\xmltag{requires}) upon other OSCAR Packages, e.g.,
Env-Switcher.  If this meta file is not included a simplistic default is
used---install all files in \directory{RPMS/} on all machines in cluster.
The available XML elements for use in this file are listed in
Appendix~\ref{sect:supported-xml-tags}, page~\pageref{sect:supported-xml-tags}
with a complete example given in Section~\ref{sect:example-pkg}.

The \xmltag{rpmlist} contains the list of RPM names (without version
numbers) to be installed for the package.  There may be multiple instances
of these \xmltag{rpmlist}'s using the \xmltag{filter} to differentiate
where necessary.  The \xmltag{filter} constraints enable an author to
express what \xmlattr{distribution}, \xmlattr{distribution\_version} and
\xmlattr{architecture} the associated \xmltag{rpm}'s support.  The
\xmlattr{group} value specifies where the to install the \xmltag{rpm}'s.
Things are implicitly global if no constraints are provided, i.e., entire
cluster for all distributions.  

% Table with all currently supported XML tags & brief description
%%  $Id: pkg-xml-table.tex,v 1.1 2003/09/08 22:31:35 naughtont Exp $

% TJN: Note I'm using the "(" symbol for the \verb delimiters b/c
%      I don't believe this symbol will occur in any of the XML elements
%      or attributes.  Any other non-occuring char could be used.
%
% TJN: I need some way to convey heirarchy, currently using multi-columns
%      xml   &      &  descr... \\
%      dtd   &      &  descr... \\
%      oscar &      &  descr... \\
%            & name &  descr... \\
%
% TJN: Somehow I need to display <tag></tag>  VS. <tag/>
%      Also, certain things like <description></description> must be
%      on same line as closing multi-line text.  Need to check this out
%      I believe it's a product of using XML::Simple.
%
% TJN: Also, I need to denote optional vs. required elements

\begin{table}[htbp]
  \begin{center}
  \begin{tabular}{|l|l|l|} \hline
  {\bfseries Top-Level Tag} & {\bfseries Sub Tag} & {\bfseries Description}\\\hline
  \hline
  \verb(<?xml version="1.0"?>( & &  Opening XML Version string         \\ \hline
  %\verb(<!DOCTYPE package SYSTEM "../package.dtd">( & & DTD Include line\\\hline

  \verb(<!DOCTYPE ...>( & & Document Type Definition (DTD) Include line\\ \hline
  \verb(<oscar>(   &   & OSCAR Package Element                         \\ \hline
  \verb(<name>(    &   & Name of the software package                  \\ \hline

  %\verb(<version>( &   & Version of the software package               \\ \hline
  \verb(<version>( &   & Version of the software package           \\\cline{2-3}
     & \verb(<major>(  & Major version of the software package     \\\cline{2-3}
     & \verb(<minor>(  & Minor version of the software package     \\\cline{2-3}
     & \verb(<subversion>( & Sub-version of the software package   \\\cline{2-3}
     & \verb(<release>(    & Release number of the software package\\\cline{2-3}
     & \verb(<epoch>(      & Release number of the software package\\\hline

  \verb(<class>(   &   & Classification of package, e.g. third-party   \\ \hline
  \verb(<summary>( &   & Brief one-line description of package         \\ \hline
  \verb(<license>( &   & Software licence for the package              \\ \hline
  \verb(<group>(   &   & RPM-style software group, e.g., Application/System\\\hline
  \verb(<url>(     &   & Homepage for software package                 \\ \hline

  %\verb(<packager>(&   & OSCAR package author                         \\ \hline
  \verb(<packager>(&   & OSCAR package author                      \\\cline{2-3}
     & \verb(<name>(   & Name of package author                    \\\cline{2-3}
     & \verb(<email>(  & Email address of package author           \\\hline

  %\verb(<maintainer>(& & OSCAR package author                         \\ \hline
  \verb(<maintainer>(& & OSCAR package author                      \\\cline{2-3}
     & \verb(<name>(   & Name of package author                    \\\cline{2-3}
     & \verb(<email>(  & Email address of package author           \\\hline

  \verb(<description>(&& Brief description of the package              \\ \hline

  %\verb(<rpmlist>(&    & List RPMS included in the rpmlist             \\ \hline
  \verb(<rpmlist>(&    & List RPMS included in the rpmlist         \\\cline{2-3}
     & \verb(<filter>( & Filter applied to all RPM in list         \\\cline{2-3}
     & \verb(<rpm>(    & Name of RPM (without version)             \\\hline

  %\verb(<filter>(&     & Filter Attributes used on RPMLIST              \\\hline
  \verb(<filter>(&     & Filter Attributes used on RPMLIST            \\\cline{2-3}
     & \verb(group=(   & Attribute specifying target location for RPMS\\
     &                 & e.g., ``oscar\_server'' or ``oscar\_clients''\\\cline{2-3}
     & \verb(distribution=(& Attribute specifying supported distribution   \\
     &                 & e.g., ``redhat'', ``mandrake'', ``suse'', ``rhas''\\
     &                 & see also: \emph{lib/OSCAR/Distro.pm}              \\\cline{2-3}
     & \verb(distribution_version=(& Attribute specifying supported distribution version  \\
     &                 & e.g., ``7.3'', ``9'', ``2.1AS''                   \\\hline

  %\verb(<requires>(&   & List requirement/dependency                    \\\hline
  \verb(<requires>(&   & List requirement/dependency                \\\cline{2-3}
     & \verb(type=(    & Tag specifying type of requirement         \\
     &                 &  e.g., ``package'' (OSCAR Package),        \\
     &                 &  \emph{``rpm'' (not implemented yet)}      \\\hline

  %\verb(<package>( &   & Package specific ODA material                \\\hline 
  \verb(<package>( &    & Package specific ODA material            \\\cline{2-3}
     & \verb(<shortcut>(& ODA shortcut defined by a package            \\\hline

  \end{tabular}
  \caption{Overview of existing XML tags for OSCAR Packages} 
  \label{tab:pkg-xml-tags}
  \end{center}  
\end{table}



\subsection{RPMS \& SRPMS}
\label{sect:pkg-rpms-srpms}

The pre-compiled binary version of the software is provided in RPM format.
The RPMS are placed, obviously enough, in the \directory{RPMS} directory.
The OSCAR Wizard copies all files from this \directory{RPMS} directory to
the \directory{/tftpboot/rpm} directory.

% % TJN: (9/26/03) I prefer this text but it is not correct! ;)
% %  Ultimately this should be closer to what actually happens.
% 
% The OSCAR Wizard copies all files listed in the \xmltag{rpmlist} from this
% \directory{RPMS} directory to the \directory{/tftpboot/rpm} directory or
% alternately the directory specified by the environment variable
% \envvar{OSCAR\_RPMPOOL}.   The \xmltag{rpmlist} that is processed, based
% upon The exact \xmltag{rpmlist} that is processed is determined based upon
% the the \xmltag{filter} that is entirely true and most particular.  The
% list of supported distributions for each OSCAR release is typically stored
% in Table 1 of the installation document.

If present, whichever \xmltag{rpmlist} that fits the available distribution
(based on \xmltag{filter} constraints) is recorded in the OSCAR database
for that particular package\footnote{The contents of \xmltag{rpmlist} have
no bearing on what is copied to \directory{/tftpboot/rpm}.  All files in
\directory{RPMS} are blindly copied.}.
The list of supported distributions for each OSCAR
release is typically stored in Table 1 of the installation document.



\begin{verse}
   {\bfseries Notice: } As of OSCAR-2.3 the \xmltag{filter} tag does not
   yet support the \xmlattr{subdir} attribute, which is used to specify a
   directory for the \xmltag{rpm}'s in the \xmltag{rpmlist}.  Due to this
   limitation, some packages use a \file{setup} script to copy the
   appropriate files to this \directory{RPMS} area based on the values
   (distro\_name, distro\_ver) returned from Perl method
   \verb=OSCAR::Distro::which_distro_server()=.  To access this method from
   your \file{scripts/setup} file add the includes: 
   \begin{footnotesize}
   \begin{verbatim}
           use lib "$ENV{OSCAR_HOME}/lib";
           use OSCAR::Distro;
           my ($distro_name, $distro_ver) = which_distro_server();
   \end{verbatim}
   \end{footnotesize}
\end{verse}




\subsection{scripts}
\label{sect:pkg-scripts}

The OSCAR framework recognizes the set of scripts outlined in
Table~\ref{tab:pkg-scripts}.  A package author may use any/all of these as
needed for application configuration.  The order of operation during the
OSCAR installation process is summarized in
Table~\ref{tab:sequence-of-events}.  The ``Location'' column indicates
where the actual modification/operations take place\footnote{Note, the
\file{post\_install} scripts in practice often operate on both the server
and client filesystems -- regardless of what the original specifications
suggested.}.  Each phase of scripts is executed per package, with packages 
processed in alph-order. 
%\footnote{The Perl function \file{readdir()} is used to build the list.}.
% TJN: Confirm this before uncommenting...have to look at current code.


% Table outlining the available OSCAR API scripts
%  $Id: pkg-scripts-table.tex,v 1.4 2003/10/06 22:26:19 naughtont Exp $


\begin{table}[htbp]
  \begin{center}
  \begin{tabular}{|c|l|l|} \hline
%%%
  {\bfseries Seq\# } & {\bfseries Script Name} 
		& {\bfseries Description} 
		\\\hline
%%%
  \hline
  1 & \file{setup}                      
		& Perform any package setup 
		\\ \hline
%
  2 & \file{pre\_configure}             
		& Prepare package configuration (dynamic user input)
		\\ \hline
%
  3 & \file{post\_configure}            
		& Process results from package configuration (user input results)
		\\ \hline
%
  4 & \file{post\_server\_rpm\_install} 
		&  Perform ``out of RPM'' operations on server (limited cluster knowledge)
		\\ \hline
%
  5 & \file{post\_client\_rpm\_install} 
		&  Perform ``out of RPM'' operations on client (limited cluster knowledge)
		\\ \hline
%
  6 & \file{post\_clients}              
		&  Perform configurations with knowledge about cluster nodes (pre node install)
		\\ \hline
%
  7 & \file{post\_install}              
		&  Perform final configurations with fully install/booted cluster nodes
		\\ \hline
%
\hline
%
    & \file{post\_server\_rpm\_uninstall}              
		&  (\emph{NEW}) Runs on server's filesystem during OSCAR package uninstall
		\\ \hline
%
    & \file{post\_client\_rpm\_uninstall}              
		&  (\emph{NEW}) Runs on client's filesystem during OSCAR package uninstall
		\\ \hline
  \end{tabular}
  \caption[Package API Scripts]{The set of available OSCAR API scripts
  listed in order of execution.}
  \label{tab:pkg-scripts}
  \end{center}  
\end{table}



\begin{table}[h!]
  \begin{center}
      \begin{tabular}{rll}
        \hline
        \multicolumn{2}{c}{Description} &
        \multicolumn{1}{c}{Location} \\
        \hline
%		
		1. & Install framework pre-requisites  & Server filesystem \\
%
      1.1. & Call the API scripts: \cmd{setup} & Server filesystem \\
%
      1.2. & Read XML config files & Server filesystem \\
%
      1.3. & Install server core RPMs & Server filesystem \\
%
      1.4. & (\emph{Optional}) Download additional packages & Server
              filesystem \\
%
    1.4.1. & \hspace{2pt} Call the API scripts: \cmd{setup} & Server 
              filesystem \\
%
    1.4.2. & \hspace{2pt} Read XML config files & Server filesystem \\
%
      1.5. & Select which packages to install & Server filesystem \\
%
      1.6. & Call the API scripts: \cmd{pre\_configure} & Server
              filesystem \\
%
      1.7. & Configure the selected packages & Server filesystem \\
%
      1.8. & Call the API scripts: \cmd{post\_configure} & Server
             filesystem \\
%
      1.9. & Install the server non-core RPMs & Server filesystem \\
%
        2. & Call the API scripts: \cmd{post\_server\_rpm\_install} &
        Server filesystem \\
%
        3. & Install all the client RPMs & 
		Client/\cmd{chroot}'d environment \\
%
        4. & Call the API scripts: \cmd{post\_client\_rpm\_install} &
        Client/\cmd{chroot}'d environment \\
%
        5. & Define clients in the OSCAR/SIS database & Server
        filesystem \\
%
        6. & Call the API scripts: \cmd{post\_clients} & Server
        filesystem \\
%
        7. & Push the images to the nodes & Server filesystem \\
%
        8. & Call the API scripts: \cmd{post\_install} & 
		   Server filesystem (access to clients)\\
        \hline
      \end{tabular}
	  \caption[Outline of operations performed]{Note the \cmd{chroot}'d
	  environment indicates the operations happen in the SIS image not on
	  the actual machine.  In step 8, the script runs on the Server but can
	  affect the clients, e.g., via C3 commands.}
    \label{tab:sequence-of-events}
  \end{center}
\end{table}

\subsubsection{Package Setup}

The \file{setup} script executes before any packages are installed.  This
phase can be used to move files around in the package's directory or to do
dynamic setup before the package \file{config.xml} scripts are processed.
Once these XML files have been processed the available packages are then
passed to the GUI where they are processed by the Selector panel.  If any
of the selected packages contain \file{pre\_configure} scripts those are
processed and then handed to the Configurator.  After the Configurator has
run any existing \file{post\_configure} scripts are processed.  At this
point no package software has been installed and all that is know by the
database is what packages were selected and their \file{config.xml}
information (version, etc.).

\begin{verse}
   {\bfseries Notice: } If a user skips the ``Configure Packages'' step in
   the Wizard, the \file{pre\_configure} and \file{post\_configure} scripts
   will not be processed, i.e., the Configurator will not be run. 
\end{verse}




\subsubsection{Configurator}
\label{sect:pkg-configurator}

Packages may obtain user input via a simple facility called the
``Configurator''.  The package author writes a simple HTML Form style
document that is presented to the user if the package is selected for
installation.  The standard multi-pick lists, radio button, checkbox fields
are available.  Typically default values are provided to simplify matters
where possible for users.  

To make use of this facility create a file in the top-level of the
package's directory called \file{configurator.html}.  After the package
selection phase of the OSCAR Wizard all packages containing this file are
processed by the Configurator.  The results of this processing are written
out in XML format to the top-level directory of the package in a file
called \file{.configurator.values}.  At this point the
\file{post\_configure} API scripts are fired so packages may read the
results of the configuration phase.    The Perl \file{XML::Simple} module
is typically used for processing these results in conjunction with the
\envvar{OSCAR\_PACKAGE\_HOME} environment variable.  Alternatively, you can
use the Perl subroutine \texttt{readInConfigValues} available in the
\file{OSCAR::Configbox} module.  A complete summary of the Configurator is
available in Section~\ref{sect:configurator}, page~\pageref{sect:configurator}.
Also, a package example containing input and simple processing scripts for
output is available in Section~\ref{sect:example-configurator},
page~\pageref{sect:example-configurator}.



\subsubsection{Fixups without RPM modification}

The \file{post\_server\_rpm\_install} and \file{post\_client\_rpm\_install}
are useful when you would like to leave an RPM untouched and perform
``fixups'' outside the RPM itself.  The actual cluster nodes have not yet
been defined so no information about number of nodes or names is available
at this phase.  As the name suggests these are performed on the server
(after all server packages have been installed) and on the client (once the
client has been ``installed''\footnote{In the case of OSCAR this means that
the SIS image (i.e., SystemImager image) has been built and client software
is installed in this \cmd{chroot}'ed environment on the server node, but is
not yet on the physical compute node hardware.  This precludes the use of
things like client specific environment variables or process managment
tools e.g., \cmd{service}.}).  This pair of scripts is pretty limited in
use but helps some instances where RPMs would otherwise have to be
modified.



\subsubsection{Setup after Clients Defined}

Once the cluster nodes have been defined the \file{post\_clients} scripts
are processed.  The number of nodes, hostname/IPs and associated image are
obtained in this ``definition'' phase.  However, the nodes themselves have
not yet been physically installed.  This \file{post\_clients} phase is when
any package that needs knowledge about cluster nodes can query for the
count, names, etc.  

\begin{verse}
   {\bfseries Notice: } The method for obtaining this information is
   currently in flux but the following will provide a Perl hash containing
   hostname,  domain name and IP for all nodes defined in the cluster, in
   numeric order.
  \begin{footnotesize}
  \begin{verbatim}
       use lib '/usr/lib/systeminstaller';
       use SystemInstaller::Machine;

       my %hash = get_machine_listing($image);   #Image can be null for default

       foreach my $key (sort numerically keys %hash) {   #Key is nodenameN
           print $hash{$key}->{HOST}, ", ";
           print $hash{$key}->{DOMAIN}, ", ";
           print $hash{$key}->{IPADDR}, "\n--------\n";
       }

       # Special sort() sub-rtn (uses $a, $b instead of @_)
       sub numerically 
       {
           $a =~ /(\d+)$/;  $A = $1;  #pickoff node num & save in local var
           $b =~ /(\d+)$/;  $B = $1;  #pickoff node num & save in local var
           ($A <=> $B);
       }
  \end{verbatim}
  \end{footnotesize}
\end{verse}



\subsubsection{Completing cluster configurations}

The final configuration script that fires is \file{post\_install}.  At this
stage the nodes are completely installed and booted.  It is typically
assumed that they are accessible via C3 commands, e.g., \cmd{cexec} --
parallel cluster execution.  Any closing modifications are performed, such as
restarting service or pushing out files, e.g., \cmd{cpush} account files.

\subsubsection{Package Uninstall}

The two ``uninstall'' scripts are run after the actual OSCAR package RPMS
have been removed.  The \file{post\_server\_} and
\file{post\_client\_rpm\_uninstall} scripts are used to clean up any
modifications or files added outside the RPM, i.e., via OSCAR API scripts.


% TJN: finish this section
\subsection{testing}
\label{sect:pkg-testing}

Tests are run for each package.  The two scripts that are available for
this testing are: \file{test\_root} and \file{test\_user}.  These testing
scripts may be written in any language so long as they are executable. When
tests are run for the cluster, all \file{test\_root} scripts are executed
which perform any root level package tests.    

\begin{verse}
   {\bfseries Notice: } There are obvious security issues with this but
   currently all operations in the cluster installation are being performed
   by {\tt root} so care is expected at all phases.  The user tests are run
   as an actual user ({\tt oscartst}) so those tests are slightly ``less''
   dangerous and therefore most packages are using \file{test\_user}.
\end{verse}


The tests typically have PBS available and most of the \file{test\_user}
scripts simply setup and run a basic PBS job for the installed software,
e.g., PVM, MPI's.  Tests making use of PBS must be submitted via the
\file{test\_user} script.  The \file{pbs\_test} helper script is used to
display results for the package's test as ``PASSED'' or ``FAILED'' based
upon return codes.  This script can be run in interactive or
non-interactive mode.  The arguments are detailed at the top of the 
\file{pbs\_test} file and by using the `\verb=--help=' option interactively.

As each \file{test\_user} script is processed, the list of nodes are passed
as command line arguments.  The following is an excerpt taken from an
example \file{test\_user} script that submites a PVM job to PBS via
\file{pbs\_script.pvm}.
% test_user oscartst 0 node1 node2 node3 node4 node5 ...
\begin{footnotesize}
\begin{verbatim}
      #!/bin/sh
      cd $HOME
      clients=`echo $@ | wc -w`    # Get number of args (nodes)

      $HOME/pbs_test $clients 1 $HOME/pvm/pbs_script.pvm "SUCCESSFUL" \
        $HOME/pvm/pvmtest 3 "PVM (via PBS)"

      exit 0
\end{verbatim}
\end{footnotesize}
(For further details see files in \file{\$OSCAR\_HOME/testing} and
\file{\$OSCAR\_HOME/lib/OSCAR/Package.pm}.)


 
\subsection{doc}
\label{sect:pkg-doc}

This directory contains supplemental documentation for the package.  There
are a few pre-defined \LaTeX\ files that may be incorporated into the
overall OSCAR documentation if the package's classification is either
\emph{core} or \emph{selected}~\footnote{That is to say the package is
included in the main distribution tarball -- not obtained via OPD.}.  These
files are: \file{install.tex}, \file{user.tex} and \file{license.tex}.  The
first is added to the overall \file{install.pdf} and contains information
related to the installation of the particular software package.  The latter
two files are incorporated into the \file{user.pdf}.   The user information
can be complete or simply pointers to obtaining more thorough documentation
for the particular package.  The license for all packages are listed in
this document based on the contents of this \file{license.tex} file.


