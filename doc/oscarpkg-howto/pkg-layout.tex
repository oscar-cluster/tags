% $Id: pkg-layout.tex,v 1.1 2003/08/22 20:35:51 naughtont Exp $

\section{Package Layout}
\label{sect:pkg-layout}

As OSCAR evolved it became obvious that the mechanism to configure and
install a cluster needed to be cleanly seperated from the software that was
to be installed.  Thus the \emph{OSCAR Package} (OPkg) was born.  The OPkg
layout is geared toward making things as simple as possible for package
authors.  The desire to keep things simple as well as the ``best
practices'' mantra of OSCAR led to the decision that at it's simplest, an
OPkg could be a simple RPM.  However, most software requires further
configuration for a cluster environment.  Additionally, OPkgs can contain
suplemental documentation and verification/testing scripts.  The following
will detain the contents of a typical OPkg and then elaborate on the
available packaging APIs.

As mentioned previously OPkgs can be as simple as a binary RPM compiled for
an OSCAR supported distribution.  


however, when they also provide supplemental documentation and a meta file
describing the package.  The packaging API provides authors the ability to
make use of scripts to configure the cluster software outside of the RPM
itself.  The scripts fire at different stages of the installation process
and test scripts can be added to verify the process. Additionally, an OSCAR
Package Downloader (OPD) (see Section~\ref{sect:opd}) is provided to
simplify acquisition of new packages.

The basic directory structure for an OSCAR Package is as follows:
\begin{quote}
\begin{description}
    \item[\file{config.xml}] -- meta file with description, version, etc.
    \item[\file{RPMS/}] -- directory containing binary RPM(s) for the package
    \item[\file{SRPMS/}] -- directory containing source RPM(s) used to build
                            the package
    \item[\file{scripts/}] --  set of scripts that run at particular times
                     during the installation/configuration of the cluster
    \item[\file{testing/}] -- unit test scripts for the package
    \item[\file{doc/}] -- documentation and/or license information
\end{description}
\end{quote}


\subsection{\file{config.xml}}
If no meta file is included with the package a simplisitc default is used.
The currently valide 

\subsection{RPMS \& SRPMS}

