% $Id: rules-of-thumb.tex,v 1.1 2004/01/22 03:42:04 naughtont Exp $

\section{Rules of Thumb}
\label{sect:rules-of-thumb}


%--------------------------------------------------------------------------
\subsection{Install Location}

The general suggestion is to install into a non-global directory location.
Typically the \directory{/opt} directory is used in OSCAR.  This was based
on recommendations from the Linux File System Hierarchy standard drafts. 
%TJN: Double check the FSH name.
When installing into non-global directories, the recommended method for
adding your application to \envvar{PATH} is via the Env-Switcher tool (aka
Switcher) Appendix \ref{sect:switcher}, \pageref{sect:switcher}.



%--------------------------------------------------------------------------
\subsection{OSCAR Specific RPMS}

When creating RPMs that are specific in some manner to OSCAR it is
generally a good idea to modify the package's ``Name:'' (within the
\file{.spec} file).  This reduces conflicts with any mainstream versions of
the package when using automated update tool for example.  The suggested
format is: \verb=<name>-oscar-<ver>-<rel>.<arch>.rpm=,  e.g.,
\verb=lam-oscar-7.0.4-1.i586.rpm=.

Another tip when making modifications to an existing RPM is to make sure
you update the ``Packager:'' information to reflect who did the
re-packaging.  Also it can be helpful to change the ``Conflicts:'' and
``Provides:'' information as follows.
\begin{itemize}

	\item Conflicts: <Mainstream Packages Name/Version>

	\item Provides: <Name of Mainstream Package/Version>

\end{itemize}
This can be used to keep the mainstream items from accidentally working
with the modified OSCAR specific versions via the ``Conflicts:''.
Additionally this lets you satisfy circular depedencies for packages you
don't rebuild via the ``Provides:''.




%--------------------------------------------------------------------------
\subsection{\directory{init.d} scripts}

It is a good idea to honor the \cmd{start}, \cmd{stop}, and \cmd{restart}
targets.  It is also helpful to have a \cmd{status} target to display
whether the given application is running currently.  Also, the scripts
should behave properly when used in conjunction with the \cmd{service}
command, e.g., \cmd{service sshd status}.  The \cmd{service} command is
used throughout the Red Hat distributions and may be part of the new LSB
drafts (\emph{not 100\% sure on LSB}).  The command runs in a stripped down
environment which sometime caused problems if assuptions are made about the
execution shell environment.




%--------------------------------------------------------------------------
\subsection{Generating Configuration Files}

Often a default configuration file is included in the RPM itself.  While
this can be helpful, it often leads to scripts that have to modify the
default config file.  It is generally cleaner to just generate the file in
one place.  

Another thing to remember is that the hostname ``oscar\_server'' should
always resolve to the internal interface of the OSCAR cluster.  This can be
helpful when generating defaults.

One other note 
