%---------------------------------------------------------------------------
% $Id: opd.tex,v 1.1 2003/08/22 20:35:51 naughtont Exp $
%
% This section is from Jeff Squyres' section of the OLS'03 paper.
%
%  Ottawa Linux Symposium 2003 (OLS'03) Paper
%  July 23-26, 2003,  Ottawa Canada
%  http://www.linuxsymposium.org/2003/
%
% Information on the OSCAR package downloader.
%
%---------------------------------------------------------------------------

The OSCAR Package Downloader (OPD) provides the capability to download
and install OSCAR software from remote package repositories.  A
package repository is simply an FTP or web site.  Given the ubiquitous
access to FTP and web servers, any organization can host their own
OSCAR package repository and publish their packages on it.  There is
no central repository; the OPD network was designed to be distributed
such that no central authority is required to publish OSCAR packages.
%
Although the OPD client program downloads an initial list of
repositories from the OSCAR Working Group web site,\footnote{The
  centralized repository list is maintained by the OSCAR working
  group.  Upon request, the list maintainers will add most repository
  sites.} arbitrary repository sites can be listed on the OPD command
line.

Since package repositories are FTP or web sites, any traditional FTP
client or web browser can also be used to obtain OSCAR packages.  Most
users prefer to use the OPD client itself, however, because it
provides additional functionality over that provided by traditional
clients.  OPD offers two interfaces: a simple menu-based mechanism
suitable for interactive use and a command-line interface suitable for
use by higher-level tools (or automated scripts).

Partially inspired by the Comprehensive Perl Archive Network (CPAN),
OPD provides the following high-level capabilities:
%
\begin{itemize}
\item Automating access to a central list of repositories
\item Browsing packages available at each repository
\item Providing detailed information about packages
\item Downloading, verifying, and extracting packages
\end{itemize}
%
While the job that OPD performs is actually fairly simple and could be
performed manually, having an automated tool for these functions
provides ease of use for the end-user, performs multiple checks to
ensure that downloaded and extracted properly, and lays the groundwork
for higher-level OSCAR package/retrieval tools.
