% $Id: intro.tex,v 1.3 2003/09/25 20:42:23 naughtont Exp $

\section{Introduction}
\label{sect:intro}

The Open Source Cluster Application Resources (OSCAR) toolkit is used to
install and maintain computing clusters.  The configuration and
installation of software is central to cluster management.  The approached
used in OSCAR is to create a ``package'', which contains the necessary
software as well as additional scripts for configuration on a standard
OSCAR cluster.  This enables software to be added to the installation
framework with as little effort as possible by the package authors.  This
document will describe what comprises an \emph{OSCAR Package}, giving
examples for clarity.  The intent is to provide a single document for those
seeking to create packages for use with the OSCAR toolkit.

\subsection{Super Short Summary}

For the truely impatient here is a very short summary of what it takes to
create an OSCAR package.  The remainder of this document will describe
things in more detail.  The items marked \emph{Optional} are just that, but
are highly recommended.

\begin{itemize}
	\item Make a directory based on the package's name, e.g., \file{pvm/}.

	\item Create the package directory structure,
	       \file{<PKG\_NAME>/\{RPMS,SRPMS,scripts,testing,doc\}}

	\item Add a \file{config.xml} meta file to the top-level package directory.

	\item Add a binary RPM to the \file{<PKG\_NAME>/RPMS} directory for 
	      supported OSCAR distro(s).

	\item (\emph{Optional}) Add the the Source RPM (SRPM) to the
	      \file{<PKG\_NAME>/SRPMS} directory

	\item (\emph{Optional}) Add any configuration scripts as per OSCAR
	      script API to \file{<PKG\_NAME>/scripts}

	\item (\emph{Optional}) Add package tests to
	\file{<PKG\_NAME>/testing}, (special: \file{test\_user, test\_root})

	\item (\emph{Optional}) Add package documentation/license to
	\file{<PKG\_NAME>/doc}, (special: \file{\{user,install,license\}.tex})

	\item Tar/gzip the entire package directory and add to an OPD
		repository\footnote{How things are added to OSCAR varies but most all
		packages can be obtained via on-line package repositories via
		OPD.}, e.g., \cmd{tar -zcvf <PKG\_NAME>.tar.gz}.
\end{itemize}

