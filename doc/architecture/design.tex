% -*- latex -*-
%
% $Id: design.tex,v 1.12 2002/03/13 05:51:07 bwbarrett Exp $
%
% $COPYRIGHT$
%

\section{Design of OSCAR 2.0}

In the first generation of OSCAR, the procedure was developed first
and the Wizard was then created to add an easy way to step through the
process. However, by doing so, there was no easy way to modify the
process without redesigning the Wizard. In order to avoid such
problems and to make OSCAR more user friendly, the architecture for
the second generation will be focused around the OSCAR Wizard. As
such, all operations and functionality available to the user will be
initiated from within the Wizard. Similarly, all interaction with the
OS installers, OSCAR Package Management, and OSCAR Data Repository
will be performed by the Wizard.

\subsection{Directory Structure}

OSCAR 2.0 and later will be installed into the \file{/opt} directory,
in accordance with the Linux Standards Base Filesystem Hierarchy (LSB
FSH).

The directory structure for OSCAR 2.0 will look like the following:

\begin{itemize}
\item \file{oscar[-<version>]/}: OSCAR base directory.  As described
  in the OSCAR Developer's Guide, this directory will have no version
  number in a developer's copy, and will have a version number in an
  OSCAR distribution package.
  
  All other directories and files listed below are in this top-level
  directory.

\item \file{core-packages/}: OSCAR core functionality packages.
  
\item \file{contrib-packages/}: OSCAR packages contributed by third
  parties.

\item \file{doc/}: HowTos and other documentation.

\item \file{oscar}: Executable to start the OSCAR Wizard.

\item \file{packages/}: certified OSCAR packages.

\item \file{COPYING}: GNU General Public License v2.0.
  
\item \file{README}: Quick overview and pointers to documentation.
\end{itemize}

Note that all three kinds of packages are identical in form and in the
API that they utilize.  The only difference between them is their
location in the OSCAR directory tree.

The three package directories are explained in the following sections.

\subsubsection{\file{core-packages}}

This directory is intended for the core functionality of OSCAR.  All
the packages in this directory will be installed regardless of the
installation mode selected by the user.

Examples of core packages include (but are not limited to): C3 Power
Tools and the System Installer Suite (SIS)


\subsubsection{\file{packages}}

Packages in this directory are likely to be authored by OCG members,
but must certainly be certified by the OSCAR group.  They are
considered part of the OSCAR distribution, and therefore fall in the
same category of quality control as the core packages.

Included packages may or may not be installed by default, depending on
the installation mode selected by the user.  See
Section~\ref{sec:design-wizard}.

Examples of included packages include (but are not limited to):
LAM/MPI, MPICH, PVM, OpenSSH, and PBS.


\subsubsection{\file{contrib-packages}}

This directory does not exist in the developer's tree because it is
exclusively intended as the location for third party developers to
place their packages for inclusion in the OSCAR framework.  Depending
on the installation mode (see Section~\ref{sec:design-wizard}), the
packages in this directory will be scanned and listed as optional
packages that may be installed by OSCAR.

That is, simply the presence of a package in this directory (and
conformance to the OSCAR package API) will enable it to be installed
and managed by the OSCAR framework.  

All packages in this directory are considered optional for OSCAR
functionality.  However, these packages may have dependencies such
that inclusion of one third party package may trigger the installation
of other third party packages.


\subsection{Node Installation}

The default OS installer in OSCAR 2.0 will be the System Installation
Suite (SIS), the collaboration between the LUI and SystemImager
projects. As such, the Wizard will be designed to work with SIS in
order to install the OS on client machines when users wish to do so.
Any other future OS installation solutions will also need to be
integrated with the Wizard.


\subsection{Software Package Management}
\label{sec:design-software-package-mgmt}

\begin{discuss}
We need to discuss this point at some point, because what we had here
really doesn't apply any more...
\end{discuss}

\subsection{Package API Requirements}

Introduced in the 1.3 release, the Package API outlines the
requirements on a package in order to be included in the OSCAR
distribution.  This API is sufficient for image-based installation and
has been designed with an eventual migration towards either
image-based installation or node-based installation in mind.

This proposal does not discuss directory locations or the layout of an
OSCAR package itself -- those details can be sorted out later.  This
proposal attempts to tackle the design issues for the major steps of
the API necessary for an OSCAR package.


\subsubsection{Goals}

The goals of this API proposal are as follows:

\begin{itemize}
\item replace the API that is used for 1.2
\item support current functionality used in all current OSCAR packages
\item make an API that is implementable by 1.3
\item make the API small
\item allow functionality for adding and deleting a node
\item allow functionality for adding a package after the initial install
\item preserve the "do no harm" mentality
\item allow for small "bite-size" increments to the API in successive
      versions, hopefully without having to change the API that is in
      place so far
\end{itemize}

Note that we're still assuming the one-head-node, many-clients model.


\subsubsection{Terminology}

The following terminology is used throughout the document:

{\bf Client filesystem}: The filesystem of the client, which may be an image
on the server, or the filesystem on the client node itself.

{\bf Client node}: Every package can partition the cluster nodes into
server(s) and clients.  In the simplest cluster, every node except the
head node is a client node for every package.

{\bf Image-based installation}: An installation where an "image" of the
node is maintained on a server.  This tree is then used to update
clients.  This is the current OSCAR installation method, implemented
by SIS.

{\bf Node-based installation}: An installation where each node is
individually installed and configured.  This is not currently
implemented by OSCAR, but is expected at a later date.

{\bf Server filesystem}: The filesystem on the server itself,
*excluding* any client images that may reside on the server.

{\bf Server node}: Every package can partition the cluster nodes into
server(s) and clients.  In the simplest cluster, the headnode is the
server for every package.


\subsubsection{Sequence of Events}

First, let's explain the sequence of events in the OSCAR installation.
This will include the proposed API calls -- they'll be explained in
more detail below.  

% BWB: FIx me!
\begin{verbatim}
   Description                                  Location
   -------------------------------------------- ------------------
1. Install all the server RPMs                  Server filesystem
2. Call the API script(s): post_server_rpm_install  Server filesystem
3. Install all the client RPMs                  Client filesystem
4. Call the API script(s): post_client_rpm_install     Client filesystem
5. Define clients in the OSCAR/SIS database     Server filesystem
6. Call the API script(s): post_clients         Server filesystem
7. Push the images to the nodes (**)            Server filesystem
8. Call the API script(s): post_install         Server filesystem
   -------------------------------------------- ------------------
\end{verbatim}

(**) Note that step 7 won't happen in a node-based install -- this
     step will be skipped.

Every OSCAR package can have zero or more of the API scripts (this is
different than what we have right now where there is only one
server\_prep script, for example).  At steps 2, 4, 6, and 8 in the
event sequence, each OSCAR package will be examined for the relevant
API scripts.  If it exists, it will be called.  If it does not exist,
that package will be skipped for that step.

Note that this order of events may be slightly different for different
forms of installation.  For example, in a kickstart-oriented
installation, step 5 (define clients in a database) may actually occur
earlier.  However, the order of execution of the API scripts will
still remain the same.


\subsubsection{Supplimental Files}

Before discussing the API calls, there are two supplemental files that
should be noted.

\noindent {\bf server.rpmlist}

This per-package file contains a list of RPM filenames (one per line) of
RPMs that are installed during step 1.  At present, the RPM files
listed in this file will be installed on the head node.  However, once
the restriction on a single head node running all servers is removed,
the server list may be installed on an arbitrary node or set of nodes
in the cluster (at the user's discretion).

Primitive conflict resolution is run on this list, checking for such
things as two versions of the same RPM, etc.  However, the majority of
burden for RPM conflict resolution is up to the package author --
don't include RPMs that will be problematic \footnote{ Perhaps
someday we can relax this constraint, but not for 1.3.  It still
adheres to the "do-no-harm" philosophy, because packages that don't
have conflicts will still be legal if we do eventually introduce
sophisticated RPM conflict resolution.}.

\noindent {\bf client.rpmlist}

Same as the server.rpmlist file, except that it contains the RPM
filenames for the RPMs installed in step 3.


\subsubsection{API calls}

So let's now discuss each of the API calls in detail.

\noindent {\bf post\_server\_rpm\_install (step 2)}

Guidelines:

\begin{itemize}
\item This script will be run on the head node.
\item This use of this API script is strongly discouraged.  The contents
  of this script should normally be contained in the \%post scripts in
  the RPMs themselves.  This API script is only here to help taking
  third party non-OSCAR-ized RPMs and graft them into the OSCAR
  framework without needing to re-make the RPM.
\item This script must be safely re-runnable; it may be run multiple times
  over the life of the cluster.
\item This script can *only* modify anything on the server filesystem, but
  should restrict itself to only things that are relevant to that
  package.
\item This script cannot modify any client filesystems.
\item This script cannot use anything from the OSCAR/SIS database,
  because there isn't much (any?) information in it yet.
\end{itemize}

Parameters:

\begin{itemize}
\item None.
\end{itemize}

\noindent {\bf post\_client\_rpm\_install (step 4)}

Guidelines:

\begin{itemize}
\item This script will be run on the client filesystem.  That is, it will
  either be chrooted to the image or run on all of the actual nodes
  themselves.
\item This use of this API script is strongly discouraged.  The contents
  of this script should normally be contained in the \%post scripts in
  the RPMs themselves.  This API script is only here to help taking
  third party non-OSCAR-ized RPMs and graft them into the OSCAR
  framework without needing to re-make the RPM.
\item This script must be safely re-runnable; it may be run multiple times
  over the life of the cluster.
\item This script can modify anything on the client filesystem, but should
  restrict itself to only things that are relevant to that package.
\item This script cannot use anything from the OSCAR/SIS database
\footnote{This may change in future versions of the API, but for now,
nothing in the OSCAR/SIS DB is available during this step.}
  The only piece of global information that is available is the
  hostname oscar\_server, which will be set correctly in the /etc/hosts
  on every node/image.
\end{itemize}


Parameters:

\begin{itemize}
\item None.
\end{itemize}

\noindent {\bf post\_clients (step 6)}

Guidelines:

\begin{itemize}
\item This script will only be run on the head node.
\item This script must be safely re-runnable; it may be run multiple times
  over the life of the cluster (e.g., when nodes are added/deleted,
  packages are added, etc.)
\item This script can modify anything on the server filesystem, but should
  restrict itself to only things that are relevant to that package.
\item This script can NOT modify anything on client filesystems.
\item This script can read any information in the OSCAR/SIS database (no
  write operations are supported).  Using the OSCAR/SIS database API
  is the only way to read the database (see below).
\end{itemize}

Parameters:

\begin{itemize}
\item None.
\end{itemize}

\noindent {\bf post\_install (step 8)}

Guidelines:

\begin{itemize}
\item This script will only be run on the head node.
\item This script must be safely re-runnable; it may be run multiple times
  over the life of the cluster (e.g., when nodes are added/deleted,
  packages are added, etc.)
\item This script can modify anything on the server filesystem, but should
  restrict itself to only things that are relevant to that package.
\item This script can NOT modify anything on client filesystems.
\item This script can read any information in the OSCAR/SIS database (no
  write operations are supported).  Using the OSCAR/SIS database API
  is the only way to read the database (see below).
\end{itemize}


Parameters:

\begin{itemize}
\item None.
\end{itemize}


\subsubsection{Notes about RPMs}

Because we anticipate eventually allowing upgrading of OSCAR packages,
OSCAR RPMs should cleanly allow "rpm -Uvh" operations.

Additionally, the \%post scripts in an RPM should be used whenever
possible instead of steps 2 and 4 (post\_server\_rpm\_install and
post\_client\_rpm\_install).  Steps 2 and 4 are intended to allow random,
non-OSCAR-ized, third-party RPMs to become OSCAR packages by adding a
script instead of needing to re-make the RPM with the appropriate
\%post scripts.

Scripts in an RPM (such as \%pre and \%post) [currently] cannot use any
OSCAR database API calls, such as obtaining the number of nodes in an
OSCAR cluster.  Future versions of this API may allow such calls, but
with the stipulation that RPMS that invoke OSCAR database API calls
depend on some OSCAR-specific RPM that provide the database API calls
themselves.


\subsubsection{Major actions and how the API is used}

\noindent {\bf Initial OSCAR install}

Run all steps -- 1 through 8 (of course, skip step 5 if using a
node-based install).

\begin{enumerate}
\item Install all the server RPMs
\item Call the API script(s): post\_server\_rpm\_install
\item Install all the client RPMs
\item Call the API script(s): post\_client\_rpm\_install
\item Define clients in the OSCAR/SIS database
\item Call the API script(s): post\_clients
\item Push the images to the nodes
\item Call the API script(s): post\_install
\end{enumerate}

\noindent {\bf Add a package after the initial OSCAR install}

Run steps 1 through 4, and then steps 6 through 8.  Step 5 is not
necessary because all the clients have already been defined --
modifying the OSCAR/SIS database is not necessary just to install a
new package.

\begin{enumerate}
\item Install all the server RPMs
\item Call the API script(s): post\_server\_rpm\_install
\item Install all the client RPMs
\item Call the API script(s): post\_client\_rpm\_install
\item $<$skipped$>$
\item Call the API script(s): post\_clients
\item Push the images to the nodes
\item Call the API script(s): post\_install
\end{enumerate}

\noindent {\bf Add a node after the initial OSCAR install}

Run steps 5 through 8.  This allows packages to reconfigure themselves
if they need to, based upon the addition of a new node in the cluster
(canonical example: PBS).

\begin{enumerate}
\item $<$skipped$>$
\item $<$skipped$>$
\item $<$skipped$>$
\item $<$skipped$>$
\item Define clients in the OSCAR/SIS database
\item Call the API script(s): post\_clients
\item Push the images to the nodes
\item Call the API script(s): post\_install
\end{enumerate}


\noindent {\bf Remove a node after the initial OSCAR install}

Run steps 5 through 8.  This allows packages to reconfigure themselves
if they need to, based upon the deletion of a new node in the cluster
(canonical example: PBS).

\begin{enumerate}
\item $<$skipped$>$
\item $<$skipped$>$
\item $<$skipped$>$
\item $<$skipped$>$
\item Define clients in the OSCAR/SIS database
\item Call the API script(s): post\_clients
\item Push the images to the nodes
\item Call the API script(s): post\_install
\end{enumerate}


\noindent {\bf Accessing the OSCAR/SIS database}

NOTE: It is required that all API scripts which need node information be
written in Perl, and use the following function to gain access to the
database.

The SystemInstaller::Machine class exports a convenience function
called get\_machine\_listing.  This can optionally take an image name as
the argument \footnote{There should be a description of how to use
this here...}

\begin{discuss}
SIS Guys - How do we import this funtion in our perl scripts?
\end{discuss}

get\_machine\_listing returns a hash.  The hash keys are the names of
all the clients in the database (optionally associated with an image
if that is specified).  The hash value for each key is a hash
reference with the following keys defined:

\begin{verbatim}
  Key     Value
  ------- -----------------------------
  HOST    hostname
  IPADDR  ip address of primary adapter
  DOMAIN  domainname
  NUMPROC number of processors on node
  ------- -----------------------------
  (other values can be added if required)
\end{verbatim}

It is recommended that all user exit scripts which need node
information be written in Perl, and use this function to gain access
to the database.

Note that this function is currently SIS-specific, and will likely
change in future versions of this API.


\subsubsection{Points still unresolved}

These issues can be addressed for API versions after 1.3.

\begin{enumerate}
\item How to do upgrades of packages?
\item How to do upgrades of the entire OSCAR cluster?
\item How to do package inter-relationships?
\item How to remove a package?
\item How to make OSCAR DB remotely available (for step 4)?

\item Will need to make OSCAR DB info available remotely (in the chroot
    jail or on the nodes) for post 2.0 for node-based installs --
    perhaps some socket-based thingy.  Punted for now 'cause we'll be
    image-based until 2.0.  Making it remotely accessible will make
    OSCAR DB info in step 4.
\end{enumerate}

% LocalWords:  Exp Filesystem LSB FSH oscar doc HowTos Docs pkgname ODR CLAMDR
% LocalWords:  readDR pl writeDR IP DDR ETMASK ONFIG CONFIG dhcp Hostname eth
% LocalWords:  FQDN OUTE NUM ROCS EAD ERSION YPE PACKAGELIST packagelist Webmin
% LocalWords:  HOSTLIST hostlist headnode hostname backend CGI Perl LUI API un
% LocalWords:  SystemImager backends uninstall Uninstallation init geiselha OPM
% LocalWords:  uninstallation tarball pkgconfig atboot Solaris pkg contrib OCG
% LocalWords:  AC EFAULT accessor ODRs integrators verbage vs admin opkgconfig
% LocalWords:  integrator txt
