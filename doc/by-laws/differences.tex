% -*- latex -*-
%
% Copyright (c) 2002-2005 The Trustees of Indiana University.  
%                         All rights reserved.
%
% $COPYRIGHT$
%

\section{Differences from OCG By-Laws}

%%%%%%%%%%%%%%%%%%%%%%%%%%%%%%%%%%%%%%%%%%%%%%%%%%%%%%%%%%%%%%%%%%%%%%%%%%%

\subsection{OSCAR Working Group Chair}

The OSCAR Working Group will have an elected chair.  The term of the
Chair is one year.  Chairs can be reelected for consecutive terms.

The Chair is responsible for all meetings of the OSCAR working group.
In practice, this typically means keeping the group functioning as a
cohesive working entity.  As with all OCG positions, the emphasis of
the entire process is on group consensus.  The Chair should actively
strive for group agreement whenever possible.  Note, however, that
disagreements and conflict are handled by the OSCAR Release Manager
and Steering Committee.

The Chair's responsibilities include:

\begin{enumerate}
\item Scheduling working group and public meetings
\item Composing and distributing agendas for meetings
\item Archiving notes from meetings
\item General OSCAR advocacy
\end{enumerate}

The Chair can delegate responsibilities as necessary.

%%%%%%%%%%%%%%%%%%%%%%%%%%%%%%%%%%%%%%%%%%%%%%%%%%%%%%%%%%%%%%%%%%%%%%%%%%%

\subsection{Release Managers}

Each major OSCAR version series will have an elected Release Manager
(RM).  The term of the release manager is from the first subrelease of
a major version (e.g., 1.1) through the first release of the next
major version (e.g., 2.0), or one year, whichever is shorter.  Release
Managers can be re-elected for consecutive terms.

The RM is generally responsible for all issues surrounding the
releases during his or her term.  As with all OCG positions, the
emphasis of the entire process is on group consensus.  The Release
Manager should actively strive for group agreement whenever possible.
When not possible, the Release Manager can make final decisions in
order to make release deadlines.

The RM's responsibilities include:

\begin{enumerate}
\item Creating and maintaing a roadmap for each version in the series.
  
\item Creating, maintaining, and enforcing the schedule of all
  releases in the series.

\item Freezing the development tree of a given version in preparation
  for release.

\item Arbitrating all issues relating to a given release.  When the
  group cannot come to consensus on an issue, the RM can call for a
  vote and/or make a final decision.

\item Authorizing a release.
  
\item Ensuring that appropriate press releases are distributed for
  each release.
\end{enumerate}

Potential release managers are nominated from current OSCAR equity
holders.  The release manager is then selected by consensus by the
OSCAR core from the pool of nominees.  It is recommended that release
managers of consecutive OSCAR version series not come from the same
organization. 

%%%%%%%%%%%%%%%%%%%%%%%%%%%%%%%%%%%%%%%%%%%%%%%%%%%%%%%%%%%%%%%%%%%%%%%%%%%

\subsection{Steering Committee}

The OSCAR steering committee fullfills essentially the same role as
the OCG steering committee.  It provides overall direction and
organizational support for the rest of the working group.  The
steering committee will usually only directly intervene in the working
group when the members are deadlocked or seriously fragmented. 

There are typically four members of the OSCAR steering committee:

\begin{itemize}
\item OSCAR working committee chair
\item The RM for the current release line
\item The RM for the next release line, if that person has been
  identified
\item The RM from the previous release line, if that person is still
  an equity holder
\end{itemize}

Exceptions may be made for the steering committee members if some of
the members are from the same organization, or if the same person
would fill multiple slots.  Exceptions must be approved by both the
existing steering committee members and the working group itself.

Motions in the steering committee are passed by majority votes.  All
other aspects of voting are the same as for the OCG.
